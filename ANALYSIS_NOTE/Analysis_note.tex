\documentclass[11pt,a4paper]{article}
\usepackage[T1]{fontenc}
\usepackage[utf8]{inputenc}
\usepackage{authblk}
\usepackage[english]{babel}
\usepackage{fancyhdr}
\usepackage{subfig}
\usepackage{floatrow}
\usepackage{float}
\usepackage{amsmath}
\usepackage{amssymb}
\usepackage{slashed}
\usepackage{graphicx}
\usepackage{todonotes}
\usepackage[toc,page]{appendix}
\usepackage{hyperref}
\usepackage{placeins}
\usepackage{cleveref}
\usepackage{multirow}
\usepackage{longtable}
\usepackage[final]{pdfpages}


\hypersetup{
	colorlinks,
	citecolor=purple,
	filecolor=black,
	linkcolor=blue,
	urlcolor=black
}
\usepackage{color}

\newcommand{\white}[1]{{\textcolor{white}{#1}}}



\newcommand*\samethanks[1][\value{footnote}]{\footnotemark[#1]}
%\title{Transition Form Factor of the $\eta^{\prime}$ Meson with CLAS12}
\title{Measurement of Cross-Sections of exclusive $\pi^{0}$ Photo-production on Hydrogen from 1.1 GeV - 5.45 GeV using $\lowercase{e}^{+}\lowercase{e}^{-}\gamma$ decay from the CLAS/g12 Data}
\date{}

\author{Michael C. Kunkel\thanks{email: m.kunkel@fz-juelich.de}\thanks{Now at Forschungszentrum J\"ulich, J\"ulich (Germany)} \\ \vspace{0.3cm} \it \qquad Old Dominion University, VA (U.S.A.) \newline \newline}


\renewcommand\Authands{, }
\fancypagestyle{firststyle}
{
	\fancyhf{}
	\renewcommand{\headrulewidth}{0pt}
	\fancyhead[R]{\small CLAS ANALYSIS XXX}
}
\newlength{\figwidth}
\setlength{\figwidth}{0.9\columnwidth}

\newlength{\qfigheight}
\setlength{\qfigheight}{0.25\textheight}

\newlength{\hfigheight}
\setlength{\hfigheight}{0.5\textheight}

\def\piz{\pi^{0}}
\def\pizT{$\pi^{0} \ $}
\def\pizDal{$\pi^{0} \rightarrow e^+e^- \gamma  $}

\def\etaT{$\eta $}
\def\etaDal{$\eta \rightarrow e^+e^- \gamma  $}

\def\omT{$\omega  $}
\def\omDal{$\omega \rightarrow e^+e^- \piz $}

\def\etaP{\eta^{\prime}}
\def\etaTP{$\eta^{\prime}  $}
\def\etaPDal{$\eta^{\prime} \rightarrow e^+e^- \gamma  $}

\def\phiT{$\phi  $}
\def\phiDal{$\phi \rightarrow e^+e^- \eta  $}
\def\phiDalT{\phi \rightarrow e^+e^- \eta  }

\def\epemT{$ e^+e^-  $}
\def\pipiT{$\pi^+\pi^-$}
\def\epem{e^+e^-}

\def\phiPR{$ep\to e'p \phi \rightarrow p e^+e^- \eta$}
\def\etaPR{$ep\to e'p \etaP \rightarrow p e^+e^- \gamma$}

%\def\grpath{figures}
\def\figures{/Users/michaelkunkel/WORK/GIT_HUB/THESIS/figures/print}
\newcommand{\abbr}[1]{\textsc{\texttt{#1}}}
\newcommand{\abbrlc}[2]{\textsc{\texttt{#1}}\texttt{#2}}
%%%% new commands and macros %%%%%%%%%%%%%%%%%%%%%%%%%%%%%%%%%%%%%%%%%%%
\newlength{\figwidth}
\setlength{\figwidth}{0.9\columnwidth}

\newlength{\qfigheight}
\setlength{\qfigheight}{0.25\textheight}

\newlength{\hfigheight}
\setlength{\hfigheight}{0.5\textheight}

\newcommand{\acro}[1]{#1\@}
\newcommand{\abbr}[1]{\textsc{\texttt{#1}}}
\newcommand{\abbrlc}[2]{\textsc{\texttt{#1}}\texttt{#2}}
\newcommand{\desg}[1]{\texttt{#1}}
\newcommand{\todo}[1]{\textbf{\uppercase{\emph{#1}}}} %\textcolor{Orange}{#1}}}

\def\g12{\emph{g12}}

\def\clas{\abbr{CLAS }}

\def\figures{figures}
\def\tablepath{../../}
\newcommand{\bank}[4]{$\mathtt{#1}^{#2}_{#3}\lbrack\mathtt{#4}\rbrack$}

\def\ith{$i$\textsuperscript{th}}



%\def\coloronline{(Color online.)\ }
\def\coloronline{}

%%% particles
\def\p{\mathrm{p}}
\def\n{\mathrm{n}}
\def\Kp{\mathrm{K}^{+}}
\def\Km{\mathrm{K}^{-}}
\def\K0{\mathrm{K}^{0}}
\def\Y{\mathrm{Y}}
\def\epos{\mathrm{e}^{+}}
\def\eneg{\mathrm{e}^{-}}
\def\gamstar{\mathrm{$\gamma$}^{*}}
\def\piup{$\pi$}
\def\gammaup{$\gamma$}
\def\um{{\text{$\mu$}}m}

\newcommand{\bra}[1]{\left<#1\right|}
\newcommand{\ket}[1]{\left|#1\right>}
\newcommand{\braket}[2]{\left<#1\middle|#2\right>}
\def\piz{$\mathrm{\pi^{0}}\ $}
\def\epem{$e^+e^-\ $}
% Document starts
\begin{document}
%	\includepdf[pages=-]{CheckList_II_final.pdf}
	\maketitle
	\thispagestyle{firststyle}
\begin{abstract}
	Photoproduction of the $\pi^0$ meson was studied using the \textsc{\texttt{CLAS}} detector at Thomas Jefferson National Accelerator Facility using tagged incident beam energies spanning the range $E_{\gamma}=$~1.1~GeV~-~5.45~GeV. The measurement is performed on a liquid hydrogen target in the reaction $\gamma p\to pe^+e^-(\gamma)$. The final state of the reaction is the sum of two subprocesses for $\pi^0$ decay, the Dalitz decay mode of $\pi^0\to e^+e^-\gamma$ and conversion mode where one photon from $\pi^0\to \gamma\gamma$ decay is converted into a $e^+e^-$ pair. This specific final state reaction avoided limitations caused by single prompt track triggering and allowed a kinematic range extension to the world data on $\pi^0$ photoproduction to a domain never systematically measured before.
	
	We report the measurement of the $\pi^0$ differential cross-sections $\frac{d\sigma}{d\Omega}$ and $\frac{d\sigma}{dt}$. The angular distributions agree well with the SAID parametrization for incident beam energies below 3~GeV, while an interpretation of the data for incident beam energies greater than 3~GeV is currently being developed.
\end{abstract}
	\newpage
	\tableofcontents
	\newpage
	\section{Introduction}\label{sec:intro}
	In hadron physics, photoproduction of single pion is essential to understand the photon-nucleon vertex. At low energies, the photon-nucleon coupling establishes excited nucleon resonances which has been at the forefront of physics ''missing resonances'' search. At high energies single pion photoproduction can be used to test predictions of Regge theory, in which recent calculations~\cite{JPAC} have shown to describe the presented data well. Furthermore, these measurements have shown that the differential cross section for single pion photoproduction at fixed c.m. angles, $\theta_{c.m.}$, where the mandelstam variables s, t, and u are large, seem to scale as $\frac{d\sigma}{dt} \sim s^{2-n}f(\theta_{c.m.})$, where $s$ and $t$ are the Mandelstam variables and $n$ is the total number of interacting elementary fields in the initial and final state of the reaction. This is predicted by the constituent counting rule~\cite{scaling1,scaling2} and exclusive measurements in $pp$ and  $\bar{p}p$ elastic scattering~\cite{scalingexp5, scalingexp7}, meson-baryon $M p$ reactions~\cite{scalingexp7}, and photoproduction $\gamma N$~\cite{scalingexp2, scalingexp3, scalingexp4, scalingexp6, scalingexp8, scalingexp9, scalingexp10, scalingexp11} agree well with this rule.
	
	This analysis note details the CLAS g12 data set, the extraction of the \pizT signal from the data, the Monte-Carlo techniques utilized for acceptance correction and the track efficiency calculation used for correcting the data. Also to be shown is the differential cross-sections through the entire beam energy range of the g12 experiment, a comparison of the differential cross-section with existing world data. 
	
	This analysis note details the analysis techniques and corrections not already discussed and approved in the g12 analysis note procedure document~\cite{g12note}. All relevant procedures described in~\cite{g12note} have been applied to this analysis. See checklist.
	
	%of $70^{\circ}$, $90^{\circ}$ and $110^{\circ}$

	\section{Data Selection and Analysis Cuts}\label{sec:evnt}
\subsection{Event Selection}\label{subsec:evnt}
	The reaction chain of interest in this analysis is:
	\begin{align}
	\gamma p \rightarrow p + x
	\end{align}
	where $x$ is reconstructed from the missing mass of $p(\gamma,p)x$ off the target proton and the tagged photon. The meson $x$ can then decay according to  
	\begin{align}
	x\rightarrow e^{+}e^{-}(\gamma)
	\end{align}
	where the decay product $\gamma$ is left undetected. For the \pizT meson the decay products, $e^+$,$e^-$,$\gamma$, can arise from two main decay branching ratios found in Tab.~\ref{tab:pi0}. The first decay of $\piz\rightarrow\gamma\gamma$ can produce electron/positron pairs via external conversion inside the hydrogen target, i.e. $\gamma\rightarrow e^+e^-$ , while the second decay $\piz\rightarrow e^+e^-\gamma$ is produced via Dalitz decays. The total sum of both branching ratios accounts for $\sim$ 99.997\% of all decays of \pizT.
	\input{tables/pi0_decay}
	\FloatBarrier
	Pions were skimmed initially because lepton identification is done at the analysis level. If the event satisfied the requirements listed in Table~\ref{tab:skim.requirements}, then all timing, momentum and vertex information was 
	outputted as well as \abbr{CC} and \abbr{EC} information. To reduce the size of the data set, a cut 
	was placed on the total missing mass of $\gamma p \to p \pi^{+} \pi^{-}$ to be less than 275~MeV. This cut was broad enough to not interfere with \pizT selection from single 
	\pizT production i.e. $\gamma p \to p \pi^{0}$ when assigned the pion the lighter mass of a electron/positron. This broad cut also does not interfere with \pizT production from 
	light meson decay, i.e $\gamma p \to p \omega \to p \pi^{+} \pi^{-} \pi^{0}$. 
		
	\begin{table}[h!]
\begin{center}
\caption[Skim requirements]{\label{tab:skim.requirements}Requirements of initial skim \vspace{0.75mm}} %\vspace{0.75mm}

\begin{tabular}{lr}

\hline
Requirement  \\
\hline
One in-time beam photon  \\ 
One proton  \\
One $\pi^+$ or \emph{``unknown''} of q$^+$ \\
One $\pi^-$ or \emph{``unknown''} of q$^-$  \\
\hline \hline
\end{tabular}


\end{center}
\end{table}
\vspace{20pt}
	\textcolor{blue}{The probability that there is a $2^{nd}$ in-time photon detected is 18\% at $E_\gamma$ = 1.1~GeV and $\sim$4\% at $E_\gamma$ = 4.4~GeV and $\sim$2\% at 5.5~GeV. Therefore the chance we choose the incorrect photon is half of these value, i.e. 9\% at $E_\gamma$ = 1.1~GeV and $\sim$2\% at $E_\gamma$ = 4.4~GeV and $\sim$1\% at 5.5~GeV. The track-efficiency corrections (Sec.~\ref{sec:results.normalization}) derived with $p\pi^+\pi^-$ also were affected by this phenomena (i.e. skimmed with only one photon) and corrected this normalization. A study was performed using all in-time photons in which the track efficiency was also calculated with all in-time photons, the difference on the reported results was negligible}
	\FloatBarrier
\subsection{Lepton Identification}
  \textcolor{blue}{Typically, electron and positron identification in the CLAS is done by using selection cuts on the energy deposited in the EC and on the minimum number 
  of photoelectrons in the CC. For electron runs, this allows to separate the scattered electron from negatively-charged pions. Since we are interested in a 
  lepton pair produced in the pi0 decay for the two-body reaction $\gamma p\rightarrow p \pi^0 \rightarrow pe^+e^- (\gamma)$gammap->p0p->pe+e- gamma, we follow a different approach based in kinematics. 
  Once the data is skimmed according to Table~\ref{tab:skim.requirements}, all tracks that were identified in the PART bank as  $\pi^+$, $\pi^-$, unknown with q$^+$, or unknown with q$^-$ 
  tracks, were assigned the electron mass for the 1-C and the 2-C kinematic fitting of the $\gamma p\rightarrow pe^+e^- (\gamma)$ final state, and the pion mass for the 4-C 
  kinematic fitting of the $\gamma p\rightarrow p\pi^+\pi^- (\gamma)$ final state, as described in Sec.~\ref{sec:analysis.fitting.topology}. The 4-C fitting was used to select and remove true $p\pi^+\pi^-$ events while 
  the 1-C and the 2-C fits were used to select only true $\gamma p\rightarrow pe^+e^- (\gamma)$ and $p\pi^0$ final states, respectively. Thus, true pion tracks who are consistent with the 
  two-pion final state were removed kinematically from our sample, while all the other tracks that were assigned the electron mass but were not true 
  electron or positron tracks were also kinematically removed from the sample by the 1-C and the 2-C fits. This approach eliminated the need to apply 
  dedicated electron/positron identification selection cuts.}
   
   
%	Lepton identification was based on conservation of energy-momentum. Once the data is skimmed according to Table~\ref{tab:skim.requirements}, all particles that were $\pi^+$, $\pi^-$, unknown with $q^+$ or unknown with $q^-$ were tentatively assigned to be electrons or positrons based on their charge. This meant that the mass term of the particle's 4-vector was set to be the mass of an electron instead of that of a pion. This technique works because the mass of the \pizT (0.135~GeV) is less than the mass of $\pi^+$ or $\pi^-$ (0.139~GeV) and by laws of conservation of energy-momentum.
\subsubsection{Lepton Triggering and Neutral Triggering}\label{sec.data.trig.lepton}
In g12, since the \abbr{CC} was filled with gas, it was possible to include the \abbr{CC} as a component of the trigger. 
There were three trigger ``bits'' used for lepton identification in g12 as listed in Table~\ref{tab:data.trig.conf.2}. Each ``bit'' used a (\abbr{EC}$\cdot$\abbr{CC}) configuration to identify leptons. The (\abbr{EC}$\cdot$\abbr{CC}) configuration required a coincidence between the electromagnetic calorimeter and the Cherenkov subsystems. This coincidence was established by using the voltage sum of the \abbr{CC} for a sector and the voltage sum of the \abbr{EC} for the same sector and comparing each sum to a preset threshold described in Table~\ref{tab:data.ecccthresh}. The \abbr{EC} voltage sum threshold comparison is done on both the \abbr{EC}$_\mathrm{inner}$ and \abbr{EC}$_{\mathrm{total}}$ which are the \abbr{EC} voltage signals for the energy deposited in the inner layer and in all layers. The labels of photon or electron specified in Table~\ref{tab:data.ecccthresh} are not actual photons or electrons, but were considered a first-order approximation for detection. The particle identification is done at the analysis level. The method for determining the (\abbr{EC}$\cdot$\abbr{CC}) does not allow for multiple lepton triggering in the same sector. Determining multiple leptons in the same sector is done at the analysis level. 

The ``bit 6'' trigger configuration, (\abbr{ST}$\cdot$\abbr{TOF})$\cdot$(\abbr{EC}$\cdot$\abbr{CC}) requires a \abbr{ST} and \abbr{TOF} coincidence previously described in~\cite{g12note} along with a coincidence between the electromagnetic calorimeter and the Cherenkov subsystems described above. The (\abbr{ST}$\cdot$\abbr{TOF}) configuration of ``bit 6'' did not have to be in the same sector as the (\abbr{EC}$\cdot$\abbr{CC}) configuration of ``bit 6''. The ``bit 11'' trigger configuration, (\abbr{EC}$\cdot$\abbr{CC})$\times$2 requires two coincidences between the electromagnetic calorimeter and the Cherenkov subsystems described above, in two different sectors. 

The ``bit 5'' trigger configuration was also established as a lepton trigger. It required \abbr{EC} hits in two sectors. The ``bit 5'' trigger configuration was also established to analyze physics involving two or more neutral particles accompanied with a charged track, such as exclusive \pizT production in which the \pizT decays via 2 photons. The method for ``bit 5'' voltage sum comparison is identical to the \abbr{EC} voltage sum of ``bit 6'' and ``bit 11''

It should be noted that none of the lepton triggers required a \abbr{MOR} signal, allowing for physics involving leptons to be measured starting from g12's lowest tagger detection value of 1.142~GeV.

For this analysis, runs which had the ``bit 6'' trigger configuration were used. To satisfy the trigger requirement in the data for photon beam energies $<3.6$~GeV, cuts were placed on the \abbr{EC} and \abbr{CC} hit quantities recorded. Since either lepton could have produced the needed hits, the cut was a permutation of both leptons, i.e.

	\begin{equation}
e^+_\mathrm{\abbr{EC}hit} \ \& \ e^+_\mathrm{\abbr{CC}hit} \  \mathrm{OR} \ e^-_\mathrm{\abbr{EC}hit} \ \& \ e^-_\mathrm{\abbr{CC}hit} \ \mathrm{OR} \ e^+_\mathrm{\abbr{EC}hit} \ \& \ e^-_\mathrm{\abbr{CC}hit} \ \mathrm{OR} \ e^-_\mathrm{\abbr{EC}hit} \ \& \ e^+_\mathrm{\abbr{CC}hit} \label{eq:ECCC}
	\end{equation}
	An analysis on the trigger efficiency for \pizT events was performed and is discussed in Sec.~\ref{sec:analysis.trigger.verify}
%\vspace{0.3cm}
\FloatBarrier
\input{tables/trigger_config_2} % label: tab:data.trig.conf.2
\FloatBarrier
\input{tables/trigger_ec_cc} % label: tab:data.ecccthresh
\FloatBarrier
	
\input{DATA/kinematic_fitting}
	\input{ANALYSIS/timing}
	\input{ANALYSIS/target_density}
	\input{ANALYSIS/flux}
	\input{ANALYSIS/track_efficiency}
    \input{simulation}

	%\input{Measurement}
	%\input{Manpower}
	\section{Results}\label{sec:results}
This section will discuss the results of the cross-sections with comparisons to previous world data, as well as a comparison of SAID fits to the previous data sets and this analysis data.
\input{RESULTS/xsection}
\input{RESULTS/staterrors}
\begin{table}[h!]
\begin{center}


\caption[Systematics]{\label{tab:systematics}Relative Systematic Uncertainty used in the $\frac{d\sigma}{d\cos\theta^{\pi^0}_{C.M.} d\phi}$ measurements. \vspace{0.75mm}}

%\begin{tabular}{c|c}
\begin{tabular}{p{5.25cm} | p{5.35cm}}
\hline
Relative Systematic & Error \\
\hline
Particle Efficiency (total) & $0.005$ \\
Sector  & $ 0.0361 + 0.0065E_{\gamma}$ \\
Flux  & $ 0.057$ \\
Missing Energy Cut  & $0.02781$ \\
2-C Fit Pull Probability & $0.0219$ \\
1-C Fit Pull Probability  & $ 0.00216 + 0.01083E_{\gamma}$ \\
4-C Fit Pull Probability  & $0.00031$ \\ 
Target  & $0.005$ \\
Branching Ratio  & $0.0037$ \\
Fiducial Cut & $0.024$ \\
$z$-vertex Cut & $0.0041$ \\
%Total & $(0.0032 +0.00051E_{\gamma} +0.000184E_{\gamma}^2)^{\frac{1}{2}}$ \\
Total & $\sqrt{(6.5 +0.52E_{\gamma} +0.16E_{\gamma}^2)\cdot10^{-3}}$ \\
\hline \hline
\end{tabular}


\end{center}
\end{table}
\vspace{20pt}
\input{RESULTS/comparision} 

	
	\newpage
	\clearpage
	\phantomsection 
	\addcontentsline{toc}{section}{BIBLIOGRAPHY}
	\bibliographystyle{unsrt}
	\bibliography{PI0}	
	
%	\newpage
%	\addcontentsline{toc}{section}{APPENDICES}
%	\let\oldaddtocontents\addtocontents \renewcommand{\addtocontents}[2]{}
%	\begin{appendices} 
%		%\input{Appendix}
%	\end{appendices}

\end{document}