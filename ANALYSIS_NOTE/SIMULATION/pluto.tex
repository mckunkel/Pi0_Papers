\subsection{PLUTO++ Event Generator}\label{sec:pluto}

Pluto~\cite{PLUTO} is a Monte-Carlo event generator designed for the study of hadronic interactions and heavy ion reactions in \abbr{HADES}, \abbr{FAIR} and upcoming \abbr{PANDA} collaborations. The versatility of Pluto enables its use as an event generator for photoproduction in \abbr{CLAS}. For hadronic interactions, Pluto can generate interactions from pion production threshold to intermediate energies of a few~GeV per nucleon. The entire software package is based on ROOT and uses ROOT's embedded C++ interpreter to control the generation of events. Programming event reaction can be set up with a few lines of ROOT macro code without detailed knowledge of programming. Some features in Pluto are, but not limited to;
\begin{itemize}
	\item Ability to generate events in phase space.
	\item Ability to generate events with a continuous bremsstrahlung photon beam.
	\item Ability to generate events weighted by a user defined $t$-slope.
	\item Ability to generate events weighted by a user defined cross-section.
	\begin{itemize}
		\item Total cross section can be inputted via functional form or histogram.
		\item Differential cross sections can be inputted via functional forms or histograms for specific beam energies up to 110 histograms relating to intervals of beam energy.
	\end{itemize}
	\item Ability to generate events that decay via already established physics parameters, i.e.~transition form factors.
	\item Ability to generate events that decay via modified established physics parameters.
	\item Ability to generate events with multiple production channels, weighted by user inputted cross-section probability.
	\item Ability to generate events with multiple decay channels, weighted by user inputted branching ratio.
	\item Ability to perform vertex smearing.
	\item Ability to create virtual detectors.
\end{itemize}

For the analysis presented in this work, Pluto was used in conjunction with known differential cross sections to verify simulation momentum smearing and tagger resolution, Sec.~\ref{sec:analysis.simsmear.verify}. Pluto was also utilized as a phase space generator in this analysis, to perform a ``tune'' on the kinematic fitter, Sec.~\ref{sec:analysis.fitting}, to calculate the acceptance corrections Sec.~\ref{sec:results.acceptance}, and to calculate the normalization Sec.~\ref{sec:results.normalization}.
