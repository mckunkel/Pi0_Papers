\subsection{Simulating the Lepton Trigger}\label{sec:analysis.accept.trigger}
During the collection process, for an event to be written by the \abbr{DAQ} it must have passed at least one of the trigger ``bits" defined in Sec.~\ref{sec.data.trig.lepton}. As discussed in Sec.~\ref{sec.data.trig.lepton}, the process of lepton triggering required a coincidence between the \abbr{EC} and the \abbr{CC} subsystems. This coincidence was established by using the voltage sum of the \abbr{CC} for a sector and the voltage sum of the \abbr{EC} for the same sector and comparing each sum to a preset threshold described in Table~\ref{tab:data.ecccthresh}. However when \abbr{GSIM} simulates tracks through the \abbr{CC} and \abbr{EC}, it does not account for the minimum voltage threshold that was required for data collection, moreover the simulation of the trigger must match the trigger efficiency discussed in Sec.~\ref{sec:analysis.trigger.verify}.

Simulation of the \abbr{CC} and \abbr{EC} trigger ``bit 6'', Sec.~\ref{sec.data.trig.lepton}, was performed by writing an algorithm that attempted to mimic the method in which triggered data was recorded. To accomplish this a modified function, written by Simeon McAleer from FSU, was written into the simulation reconstruction algorithm. The routine returned the sector and a boolean of 0 or 1 (pass or fail), that simulated the trigger based on the following criteria;
\begin{enumerate}\label{trig:sim.all}
	\item The sector with the highest EC summed energy over threshold. \label{trig:sim.ECtot} 
	\item The sector with the highest EC Inner Layer summed energy over threshold. \label{trig:sim.ECinner} 
	\item The sector with the highest CC summed energy over threshold. \label{trig:sim.CCtot} 
	\item All three above conditions must be in same sector.
\end{enumerate}
Thresholds as described in Table~\ref{tab:data.ecccthresh} are 80~mV, 60~mV and 20~mV for \abbr{EC} \emph{inner}, \abbr{EC}\emph{total} and CC respectively. The \abbr{CC} trigger threshold was applied to groups of eight \abbr{CC} \abbr{PMT}s, called ``sim bits''. The ``sim bits'' were staggered by four \abbr{PMT}s so that each \abbr{PMT} goes into two ``sim bits'', after which all ``sim bits'' were ``\emph{OR}'''d together. If any ``sim bit'' calculated as above threshold, that specific sector was then compared to the remaining sectors to establish the condition listed in~\ref{trig:sim.CCtot}.

The \abbr{EC} \emph{inner} and \abbr{EC} \emph{total} trigger thresholds were applied to all \abbr{EC} strips in a sector. This was done by summing over the energy for every strip in every orientation of the \abbr{EC} per sector. If the energy summation for the \abbr{EC} \emph{inner} was above threshold,   that specific sector was then compared to the remaining sectors to establish the condition listed in~\ref{trig:sim.ECinner}. If the energy summation for the \abbr{EC} \emph{total} was above threshold, that specific sector was then compared to the remaining sectors to establish the condition of the sector with the highest EC summed energy over threshold.
\subsubsection{Validity of Trigger Simulation}
The actual triggered data could have been triggered by the following sceneries;
\begin{enumerate}\label{trig:get.all}
	\item $e^-$ \abbr{CC} and \abbr{EC} hit above preset thresholds,
	\item $e^+$ \abbr{CC} and \abbr{EC} hit above preset thresholds,
	\item $e^-$ \abbr{CC} hit above preset thresholds and $e^+$ \abbr{EC} hit above preset thresholds in the same sector, 
	\item $e^-$ \abbr{EC} hit above preset thresholds and $e^+$ \abbr{CC} hit above preset thresholds in the same sector. 
\end{enumerate}
The lepton trigger ``bit 6" was 100\% efficient (see Sec.~\ref{sec:analysis.trigger.verify}) when the data was cut using all the conditions listed above (1, 2, 3, 4) using an ``OR" flag. This means that a $\gamma p \to p e^+ e^-$ event must satisfy at least one of the listed conditions. The reduction in events when at least one of the conditions was satisfied was 69.91\%. Prior to simulating the trigger, cutting the \abbr{MC} with the listed conditions reduced the event yield by 81.91\%. Simulating the trigger and cutting on the \abbr{MC} events with the listed conditions reduced that event yield to 69.48\%. This indicates that the trigger simulation is properly mimicking the trigger configuration used when data is collected. 

%actual physics events recorded by \abbr{CLAS}.
%
%
%When all the conditions listed above are compared together using an ``\emph{OR}'' flag, on \pizT data, 69.91\% of events remain. To check the validity of the trigger simulation, events from the \pizT reconstructed simulation were placed under the conditions as the actual data. Without placing the boolean of 1 on the simulation, 81.91\% of events remain. Placing the boolean of 1 on the simulation, 69.48\% of events remain, indicating the trigger simulation is mimicking the actual physics events recorded by \abbr{CLAS}. 
\subsection{Simulation Kinematic Variables Verification}\label{sec:analysis.simsmear.verify}

In the Sec.~\ref{sec:analysis.accept.verify} the simulation was verified for efficiency. Another systematic check on the simulation was performed to investigate the validity of the kinematic variables outputted from the simulation package~\ref{sec:analysis.simulation}. This procedure was also performed as a means to double check the conclusion about the simulation efficiency found in Sec.~\ref{sec:analysis.accept.verify} and to verify whether \abbr{GSIM} simulates \emph{pair-production} properly. To perform this check, first the total number of expected \pizT events as well as $\pi^+\pi^-$ events were calculated for the beam energy range 1.1~GeV-2.8~GeV using the total cross-section, $\sigma$, for \pizT and $\pi^+\pi^-$ production found in~\cite{durham} and using
\begin{align}
N_{events} = \sigma \rho L \ ,
\end{align}
where $\rho$ and $L$ are the target density and photon flux respectively. The total number of \pizT and $\pi^+\pi^-$ events can be seen in Fig.~\ref{fig:simsmear.Ntot}, where the left axis depicts the number of \pizT events and the right axis depicts the number of $\pi^+\pi^-$ events.
\begin{figure}[h!]\begin{center}
		\includegraphics[width=\figwidth,height= 0.75 \hfigheight]{\figures/simulation/N_events.pdf}
		\caption[Total Number of \pizT and $\pi^+\pi^-$ Events Expected Between $E_{\gamma}$ 1.1~GeV-2.8~GeV ]{\label{fig:simsmear.Ntot}Total Number of \pizT (black open circles) and $\pi^+\pi^-$ (black closed triangles) events expected between $E_{\gamma}$ 1.1~GeV-2.8~GeV. The left axis depicts the events expected for \pizT production while the right axis depicts the events expected from $\pi^+\pi^-$ production. }
	\end{center}\end{figure} 
	
	Once the total amount of \pizT was determined, it was necessary to determine the amount of \pizT$\to \gamma \gamma$ and \pizT $\to e^+e^- \gamma$ to generate. This was done via the branching ratios of \pizT decay. The \pizT $\to \gamma \gamma$ has a branching ratio of $98.823 \pm 0.034$\% while \pizT$\to e^+e^- \gamma$ has a branching ratio of $1.174 \pm 0.035$\% ~\cite{pdg2014} which lead to $7.03914 \cdot10^8$ \pizT $\to \gamma \gamma$ events generated and $8.36237 \cdot10^6$ \pizT$\to e^+e^- \gamma$ events generated. Moreover, once the total amount of events were determined, the generation of the events was weighted using the \pizT differential cross-section found in the SAID~\cite{SAID} database and the $\pi^+\pi^-$ differential cross-section found in the Durham~\cite{durham} database. After the events were generated, they were processed using the simulation package described in~\ref{sec:analysis.simulation} in which afterward were given the same fiducial cuts described in Sec.~\ref{sec:analysis.data.reduction}, kinematic constraint cuts Sec.~\ref{sec:analysis.fitting.compare} and trigger simulation cuts Sec.~\ref{sec:analysis.accept.trigger}. 
	
	In Figs.~\ref{fig:simsmear.beam},~\ref{fig:simsmear.prot},~\ref{fig:simsmear.Ep},~\ref{fig:simsmear.Em} it is shown that the simulation procedure appears to give an accurate representation of physics events for the incident beam, detected proton positron and electron within \abbr{CLAS}. Furthermore, the overall acceptance and simulation of \emph{pair-production} is within a normalization factor of 1.011, meaning that the number of generated events was correct within 1.1\% or the simulation has an acceptance inefficiency of 1.1\%. The different sources contributing to the final detected \epemT topology can be seen in Fig.~\ref{fig:simsmear.EpEm}. 
	
	\begin{figure}[h!]\begin{center}
			\subfloat[$\mathrm{M_x^2(p)}$ vs. $\mathrm{M_E^2(pe^+e^-)}$ for Data][]{ %Feynman diagram of \pizT two photon decay
				\includegraphics[width=\figwidth,height=\qfigheight]{\figures/simulation/ME_vs_MxP.pdf}\label{fig:simsmear.mEMxP.data}
			}
			
			\subfloat[$\mathrm{M_x^2(p)}$ vs. $\mathrm{M_E^2(pe^+e^-)}$ for \abbr{MC}]{ %Feynman diagram of \pizT Dalitz decay
				\includegraphics[width=0.8\columnwidth,height=\qfigheight]{\figures/simulation/ME_vs_MxP_simulation.pdf}\label{fig:simsmear.mEMxP.MC}
			}
			\caption[$\mathrm{M_x^2(\gamma p \to p X)}$ vs. $\mathrm{M_E^2(\gamma p \to pe^+e^- X)}$ for simulation systematic check]{\label{fig:simsmear.mEMxP.data.MC}$\mathrm{M_x^2(\gamma p \to p X)}$ vs. $\mathrm{M_E^2(\gamma p \to pe^+e^- X)}$ for simulation systematic check. Top panel depicts data, while the bottom panel depicts \abbr{MC}.}
			
		\end{center}\end{figure}
		%
		%
		\begin{figure}[h!]\begin{center}
				\includegraphics[width=\figwidth,height= 0.75 \hfigheight]{\figures/simulation/Beam_Kinematics_fitted.pdf}
				\caption[Number of events vs. beam momentum for simulation systematic check]{\label{fig:simsmear.beam}Number of events vs. beam momentum for simulation systematic check. Comparison of incident photon beam kinematics for \abbr{MC} (black) events and data (red) when generating \abbr{MC} via differential cross-sections. Normalization factor is 1.011.}
			\end{center}\end{figure} 
			%
			%
			\begin{figure}[h!]\begin{center}
					\includegraphics[width=\figwidth,height= 0.75 \hfigheight]{\figures/simulation/Proton_Kinematics_fitted.pdf}
					\caption[Number of events vs. proton momentum (top), proton $\theta$ (middle) and proton $\phi$ kinematics for \abbr{MC} (black) events and data (red) when generating \abbr{MC} via differential cross-sections]{\label{fig:simsmear.prot}Number of events vs. proton momentum (top), proton $\theta$ (middle) and proton $\phi$ kinematics for \abbr{MC} (black) events and data (red) when generating \abbr{MC} via differential cross-sections. Normalization factor is 1.011. }
				\end{center}\end{figure} 
				%
				%
				\begin{figure}[h!]\begin{center}
						\includegraphics[width=\figwidth,height= 0.75 \hfigheight]{\figures/simulation/Positron_Kinematics_fitted.pdf}
						\caption[Number of events vs. positron momentum (top), positron $\theta$ (middle) and positron $\phi$ kinematics for \abbr{MC} (black) events and data (red) when generating \abbr{MC} via differential cross-sections]{\label{fig:simsmear.Ep}Number of events vs. positron momentum (top), positron $\theta$ (middle) and positron $\phi$ kinematics for \abbr{MC} (black) events and data (red) when generating \abbr{MC} via differential cross-sections. Normalization factor is 1.011.}
					\end{center}\end{figure} 
					%
					%
					\begin{figure}[h!]\begin{center}
							\includegraphics[width=\figwidth,height= 0.75 \hfigheight]{\figures/simulation/Electron_Kinematics_fitted.pdf}
							\caption[Number of events vs. electron momentum (top), electron $\theta$ (middle) and electron $\phi$ kinematics for \abbr{MC} (black) events and data (red) when generating \abbr{MC} via differential cross-sections]{\label{fig:simsmear.Em}Number of events vs. electron momentum (top), electron $\theta$ (middle) and electron $\phi$ kinematics for \abbr{MC} (black) events and data (red) when generating \abbr{MC} via differential cross-sections. Normalization factor is 1.011. }
						\end{center}\end{figure} 
						%
						%
						\begin{figure}[h!]\begin{center}
								\includegraphics[width=\figwidth,height= 0.75 \hfigheight]{\figures/simulation/EpEm_Sources_fitted_combined.pdf}
								\caption[Number of events vs. \epemT mass distribution for all \abbr{MC} (black) events and data (red)]{\label{fig:simsmear.EpEm}Top Panel: Number of events vs. \epemT mass distribution for all \abbr{MC} (black) events and data (red). Bottom Panel: Number of events vs. \epemT mass distribution showing the sources of the \abbr{MC} \epemT topology overlaid to the data. Normalization factor is 1.011.}
							\end{center}\end{figure} 
							
							\FloatBarrier