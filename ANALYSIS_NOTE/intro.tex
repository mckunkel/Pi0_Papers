\section{Introduction}\label{sec:intro}
	In hadron physics, photoproduction of single pion is essential to understand the photon-nucleon vertex. At low energies, the photon-nucleon coupling establishes excited nucleon resonances which has been at the forefront of physics ''missing resonances'' search. At high energies single pion photoproduction can be used to test predictions of Regge theory, in which recent calculations~\cite{JPAC} have shown to describe the presented data well. Furthermore, these measurements have shown that the differential cross section for single pion photoproduction at fixed c.m. angles, $\theta_{c.m.}$, where the mandelstam variables s, t, and u are large, seem to scale as $\frac{d\sigma}{dt} \sim s^{2-n}f(\theta_{c.m.})$, where $s$ and $t$ are the Mandelstam variables and $n$ is the total number of interacting elementary fields in the initial and final state of the reaction. This is predicted by the constituent counting rule~\cite{scaling1,scaling2} and exclusive measurements in $pp$ and  $\bar{p}p$ elastic scattering~\cite{scalingexp5, scalingexp7}, meson-baryon $M p$ reactions~\cite{scalingexp7}, and photoproduction $\gamma N$~\cite{scalingexp2, scalingexp3, scalingexp4, scalingexp6, scalingexp8, scalingexp9, scalingexp10, scalingexp11} agree well with this rule.
	
	This analysis note details the CLAS g12 data set, the extraction of the \pizT signal from the data, the Monte-Carlo techniques utilized for acceptance correction and the track efficiency calculation used for correcting the data. Also to be shown is the differential cross-sections through the entire beam energy range of the g12 experiment, a comparison of the differential cross-section with existing world data. 
	
	This analysis note details the analysis techniques and corrections not already discussed and approved in the g12 analysis note procedure document~\cite{g12note}. All relevant procedures described in~\cite{g12note} have been applied to this analysis. See checklist.
	
	%of $70^{\circ}$, $90^{\circ}$ and $110^{\circ}$
