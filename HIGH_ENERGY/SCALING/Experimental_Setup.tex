\section{\label{sec:level1}Experimental Setup}

The measurements used for this analysis was taken with the \abbr{CLAS} detector at Hall B at the Thomas Jefferson National Accelerator Facility \abbr{TJNAF} located in Newport News, Virginia. The \g12 experiment is a photoproduction experiment, it ran during March - June 2008 with a total of 44 days of good beam time. It collected over 128~TB of raw data that consisted of $26\cdot 10^9$ events, with an integrated luminosity of 68~pb$^{-1}$. The photon beam was produced by impinging a $5.715$~GeV electron beam, at 65nA, on a Au radiator of $10^{-4}$ radiation length. Photons in the energy range from 20\% to 95\% of the electron beam
energy were tagged, resulting in a photon beam energy range of 1.1-5.5~GeV. This photon beam was then collimated before being introduced onto a $\ell H_2$ target 40~cm in length along the z-direction and 2~cm radius. The placement of the target was 90~cm upstream from \abbr{CLAS} center (toward Au radiator), this increased the acceptance of particles in the forward direction. During the runtime of \g12, the Cherenkov detectors were filled with perfluorobutane ($\mathrm{C_4F_{10}}$) allowing for electron/positron detection. The experiment had a dedicated trigger, amongst 9 other triggers, that consisted of \abbr{CC} and \abbr{EC} coincidence hits for the entire beam energy range. With proper cuts on the \abbr{CC} and \abbr{EC} a $\pi/e$ rejection of $10^6$ for $e^{\pm}$ pairs was established.


		
		\subsection{\g12 Run Summary}\label{sec:clas.g12.runs}
		
		The \g12 experiment was divided into 626 production runs, 37 single-prong runs, 13 special calibration runs and numerous diagnostic runs which were not recorded. Each run consisted of approximately 50 million triggered events. If a run did not have at least 1M triggered events or if the run was corrupt, the run was discarded.
		The \g12 experiment had several special calibration runs. These runs consist of normalization, zero-field, and empty-target data runs. The normalization runs were used to calibrate the tagger for the measurement of the total photon flux and consistency of the left and right \abbr{TDC} signals of the tagger. The zero-field data was taken with the main torus magnet off. This was done to account for the position and orientation of the drift-chambers in the reconstruction. The empty target runs were used to investigate the contributions of the target wall to the data sample.

