\subsection{Lepton Triggering and Neutral Triggering}\label{sec.data.trig.lepton}
		In \g12, since the \abbr{CC} was filled with gas, it was possible to include the \abbr{CC} as a component of the trigger. 
		There were three trigger ``bits'' used for lepton identification in \g12 as listed in Table~\ref{tab:data.trig.conf.2}. Each ``bit'' used a (\abbr{EC}$\cdot$\abbr{CC}) configuration to identify leptons. The (\abbr{EC}$\cdot$\abbr{CC}) configuration required a coincidence between the electromagnetic calorimeter and the Cherenkov subsystems. This coincidence was established by using the voltage sum of the \abbr{CC} for a sector and the voltage sum of the \abbr{EC} for the same sector and comparing each sum to a preset threshold described in Table~\ref{tab:data.ecccthresh}. The \abbr{EC} voltage sum threshold comparison is done on both the \abbr{EC}$_\mathrm{inner}$ and \abbr{EC}$_{\mathrm{total}}$ which are the \abbr{EC} voltage signals for the energy deposited in the inner layer and in all layers. The labels of photon or electron specified in Table~\ref{tab:data.ecccthresh} are not actual photons or electrons, but were considered a first-order approximation for detection. The particle identification is done at the analysis level. The method for determining the (\abbr{EC}$\cdot$\abbr{CC}) does not allow for multiple lepton triggering in the same sector. Determining multiple leptons in the same sector is done at the analysis level. 
		
		The ``bit 6'' trigger configuration, (\abbr{ST}$\cdot$\abbr{TOF})$\cdot$(\abbr{EC}$\cdot$\abbr{CC}) had a \abbr{ST} and \abbr{TOF} coincidence. This requirement was to establish a track to have coincidence in one sector between any one of the four start counter paddles of that sector, and any one of the 57 time-of-flight paddles in the same sector. The (\abbr{ST}$\cdot$\abbr{TOF}) configuration of ``bit 6'' did not have to be in the same sector as the (\abbr{EC}$\cdot$\abbr{CC}) configuration of ``bit 6''. The ``bit 11'' trigger configuration, (\abbr{EC}$\cdot$\abbr{CC})$\times$2 requires two coincidences between the electromagnetic calorimeter and the Cherenkov subsystems described above, in two different sectors. 
		
		The ``bit 5'' trigger configuration was also established as a lepton trigger. It required \abbr{EC} hits in two sectors. The ``bit 5'' trigger configuration was also established to analyze physics involving two or more neutral particles accompanied with a charged track, such as exclusive \piz production in which the \piz decays via 2 photons. The method for ``bit 5'' voltage sum comparison is identical to the \abbr{EC} voltage sum of ``bit 6'' and ``bit 11''
		
		It should be noted that none of the lepton triggers required a \abbr{MOR} signal, allowing for physics involving leptons to be measured starting from \g12's lowest tagger detection value of 1.142~GeV.
		
%		\input{tables/trigger_config_1} % label: tab:data.trig.conf.1
%		
		\input{tables/trigger_config_2} % label: tab:data.trig.conf.2
%		
%		\input{tables/trigger_config_3} % label: tab:data.trig.conf.3
%		
%		\input{tables/trigger_mor.tex} % label:  tab:data.trig.mor
%		
		\input{tables/trigger_ec_cc} % label: tab:data.ecccthresh
		