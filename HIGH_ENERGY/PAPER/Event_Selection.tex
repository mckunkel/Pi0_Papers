\section{Event Selection}\label{sec:evnt}
	Pions were skimmed initially and then re-identified as leptons by changing the mass of the pion. This method is sufficient when the decaying particle's mass, i.e. $m_{\pi^0}$, is less than that of pions. If the event satisfied the requirements listed in Table~\ref{tab:skim.requirements}, then all \abbr{TOF}, \abbr{ST}, momentum and vertex information was outputted as well as \abbr{CC} and \abbr{EC} information for the $\pi^{\pm}$ particles to be used to identify leptons, as discussed in Sec~\ref{sec:analysis.pid}. To reduce the size of the data set, a cut was placed on the total missing mass of $\gamma p \to p \pi^{+} \pi^{-}$ to be less than 275~MeV. This cut was broad enough to not interfere with \piz selection from single \piz production i.e. $\gamma p \to p \pi^{0}$ when assigned the pion the lighter mass of a electron/positron. This broad cut also does not interfere with \piz production from light meson decay, i.e $\gamma p \to p \omega \to p \pi^{+} \pi^{-} \pi^{0}$. 
	\begin{table}[h!]
\begin{minipage}{\textwidth}
\begin{center}
\begin{singlespacing}
\caption[Skim requirements]{\label{tab:skim.requirements}Requirements of initial skim \vspace{0.75mm}} %\vspace{0.75mm}

\begin{tabular}{lr}

\hline
Requirement & \quad \quad Section Discussed \\
\hline
One in-time beam photon &  Sec.~\ref{sec:analysis.beam} \\ 
One proton & Sec.~\ref{sec:data.cook} \\
One $\pi^+$ or \emph{``unknown''} of q$^+$ & Sec.~\ref{sec:data.cook} \\
One $\pi^-$ or \emph{``unknown''} of q$^-$ & Sec.~\ref{sec:data.cook} \\
\hline \hline
\end{tabular}

\end{singlespacing}
\end{center}
\end{minipage}
\end{table}
\vspace{20pt}