\section{\label{sec:analysis.simulation}Monte-Carlo}
	There are certain kinematic regions of \abbr{CLAS} in which physics events are not being recorded properly i.e.~the area dividing each sector in \abbr{CLAS}. Furthermore each sector in \abbr{CLAS} is asymmetric in the acceptance of events due to subsystem inefficiencies such as inoperable \abbr{DC} wires, \abbr{PMT} inefficiencies, dead scintillator strips the the \abbr{TOF} and \abbr{ST} subsystems. When a triggered event is recorded and reconstructed these asymmetric inefficiencies factors are reflected and must be carefully understood because these factors are properties of the \abbr{CLAS} detector and independent of any physics that occurred. To properly understand the detector effects on the data, \abbr{CLAS} utilizes a \abbr{GEANT} simulation package know as \abbr{GSIM}. To prepare an event for \abbr{GSIM} the program \abbr{GAMP2PART} converts a text file, containing the 4-momentum of the generated event, into a suitable file format for \abbr{GSIM}. \abbr{GSIM} then simulates the passage of these particles through the \abbr{CLAS} detector and generates the associated \abbr{ADC} and \abbr{TDC} information from detector hits. \abbr{GSIM} takes into account detector inefficiencies described in the \abbr{\texttt{CLAS\_CALDB\_RUNINDEX}}. The \abbr{CLAS\_CALDB\_RUNINDEX} is an array of information about each subsystem's inefficiency that was derived during the \g12 calibration process. The \abbr{GSIM} simulated hits are then ``post-processed'' by smearing the \abbr{TDC} and \abbr{ADC} hits to imitate the observed resolution of the detector subsystems using the program \abbr{GPP} (\abbr{GSIM} post-processor). \abbr{GPP} also removes detector hits due to inefficient \abbr{DC} wires.
	
	 The acceptance for the cross-sections presented in this work was measured using phase-space Monte-Carlo (\abbr{MC}\label{abbr:mc}) simulation, using PLUTO++~\cite{PLUTO} as the generator, for the reaction channels,
	\begin{subequations}
		\begin{align}
				\gamma \p \rightarrow & \ p \pi^{0} \nonumber \\[-0.4em]
				& \ \hookrightarrow p \gamma \gamma \nonumber \\[-0.4em]
				& \qquad \hookrightarrow p e^+ e^- \gamma
				\label{eq:simproduct1}
			\end{align}
			\begin{align}
				\gamma \p \rightarrow & \ p \pi^{0} \nonumber \\[-0.4em]
				& \ \hookrightarrow p e^+ e^- \gamma.
				\label{eq:simproduct2}
			\end{align}
		\end{subequations} 
		A total of events, $N_g$, was generated and this number was weighted by the relative branching ratios found in Table~\ref{tab:targetspecs} to resemble the conditions of the data. The number of events generated for the reaction channel~\ref{eq:simproduct1} can be found in Table~\ref{tab:simnumspecs} as $N_c$. The number of events generated for the reaction channel n~\ref{eq:simproduct2} can be found in Table~\ref{tab:simnumspecs} as $N_d$.
		\begin{table}[h!]
\begin{minipage}{\textwidth}
\begin{center}
\begin{singlespacing}

\caption[Generated Quantities]{\label{tab:simnumspecs}Number of generated events in each decay spectrum}

\begin{tabular}{c|c|c}

%\hline \hline
%
%operation & \multicolumn{3}{c}{Generation} \\
%charge & I & II & III \\
\hline
Quantity & Value & Description\\
\hline

$N_g$ & $2.39869 \cdot 10^9$ & Total number of \piz events generated \\
$N_c$ & $2.37039 \cdot 10^9$ &  Total number of \piz $\rightarrow \gamma \gamma$ events generated\\
$N_d$ & $2.80647 \cdot 10^7$ & Total number of \piz $\rightarrow e^+ e^- \gamma$ generated\\
\hline \hline
\end{tabular}

\end{singlespacing}
\end{center}
\end{minipage}
\end{table}
\vspace{20pt}
		After the events are generated, they are inputted into the \abbr{CLAS} simulation chain \abbr{GAMP2BOS}\label{abbr:gamp2bos}, \abbr{GSIM}\label{abbr:gsim}, \abbr{GPP}\label{abbr:gpp}, and then reconstructed with the same program used to reconstruct the data, \abbrlc{}{a1c}\label{abbr:a1c}, all programs in the simulation chain use the parameters and the run index described above. Once the events are processed through \abbrlc{}{a1c}, the cuts described in Secs.~\ref{sec:analysis.data.reduction}, ~\ref{sec:analysis.fitting.compare}, are applied as they are to the real data. The acceptance $\eta(E_\gamma,\theta^{\pi^0}_{C.M.})$ is then determined by adding the simulations for the conversion and the dalitz, then for photon energy bins of 25 MeV increments and $\Delta\cos\theta^{\pi^0}_{C.M.} = 0.0125$ increments, the ratio of reconstructed events ($N_R$) to generated events ($N_G$) yields,
		\begin{equation}\label{eq:acceptance}
		\eta(E_\gamma,\cos\theta^{\pi^0}_{C.M.}) = \frac{N_R(E_\gamma,\cos\theta^{\pi^0}_{C.M.})}{N_G(E_\gamma,\cos\theta^{\pi^0}_{C.M.})} \ .
		\end{equation}
		The $\Delta\cos\theta^{\pi^0}_{C.M.}$ binning in the acceptance is a factor of 2.4 finer than the smallest $\Delta\cos\theta^{\pi^0}_{C.M.}$ increment used in the cross-section measurement. If an accurate physics model for the generator had been used, as was in Sec.~\ref{sec:analysis.simulation}, the binning for the acceptance  would not have had to be so fine.
	
	
%	\subsection{Simulating the Lepton Trigger}\label{sec:analysis.accept.trigger}
%		During the collection process, for an event to be written by the \abbr{DAQ} it must have passed at least one of the trigger ``bits" defined in Sec.~\ref{sec:clas.g12.conditions.data}. As discussed in Sec.~\ref{sec.data.trig.lepton}, the process of lepton triggering required a coincidence between the \abbr{EC} and the \abbr{CC} subsystems. This coincidence was established by using the voltage sum of the \abbr{CC} for a sector and the voltage sum of the \abbr{EC} for the same sector and comparing each sum to a preset threshold described in Table~\ref{tab:data.ecccthresh}. However when \abbr{GSIM} simulates tracks through the \abbr{CC} and \abbr{EC}, it does not account for the minimum voltage threshold that was required for data collection, moreover the simulation of the trigger must match the trigger efficiency discussed in Sec.~\ref{sec:analysis.trigger.verify}.
%		
%		Simulation of the \abbr{CC} and \abbr{EC} trigger ``bit 6'', Sec.~\ref{sec.data.trig.lepton}, was performed by writing an algorithm that attempted to mimic the method in which triggered data was recorded. To accomplish this a modified function, written by Simeon McAleer from FSU, was written into the simulation reconstruction algorithm. The routine returned the sector and a boolean of 0 or 1 (pass or fail), that simulated the trigger based on the following criteria;
%		\begin{enumerate}\label{trig:sim.all}
%			\item The sector with the highest EC summed energy over threshold. \label{trig:sim.ECtot} 
%			\item The sector with the highest EC Inner Layer summed energy over threshold. \label{trig:sim.ECinner} 
%			\item The sector with the highest CC summed energy over threshold. \label{trig:sim.CCtot} 
%			\item All three above conditions must be in same sector.
%		\end{enumerate}
%		Thresholds as described in Table~\ref{tab:data.ecccthresh} are 80~mV, 60~mV and 20~mV for \abbr{EC} \emph{inner}, \abbr{EC}\emph{total} and CC respectively. The \abbr{CC} trigger threshold was applied to groups of eight \abbr{CC} \abbr{PMT}s, called ``sim bits''. The ``sim bits'' were staggered by four \abbr{PMT}s so that each \abbr{PMT} goes into two ``sim bits'', after which all ``sim bits'' were ``\emph{OR}'''d together. If any ``sim bit'' calculated as above threshold, that specific sector was then compared to the remaining sectors to establish the condition listed in~\ref{trig:sim.CCtot}.
%		
%		The \abbr{EC} \emph{inner} and \abbr{EC} \emph{total} trigger thresholds were applied to all \abbr{EC} strips in a sector. This was done by summing over the energy for every strip in every orientation of the \abbr{EC} per sector. If the energy summation for the \abbr{EC} \emph{inner} was above threshold,   that specific sector was then compared to the remaining sectors to establish the condition listed in~\ref{trig:sim.ECinner}. If the energy summation for the \abbr{EC} \emph{total} was above threshold, that specific sector was then compared to the remaining sectors to establish the condition of the sector with the highest EC summed energy over threshold.
%		
%		\subsubsection{Validity of Trigger Simulation}
%			The actual triggered data could have been triggered by the following sceneries;
%			\begin{enumerate}\label{trig:get.all}
%				\item $e^-$ \abbr{CC} and \abbr{EC} hit above preset thresholds,
%				\item $e^+$ \abbr{CC} and \abbr{EC} hit above preset thresholds,
%				\item $e^-$ \abbr{CC} hit above preset thresholds and $e^+$ \abbr{EC} hit above preset thresholds in the same sector, 
%				\item $e^-$ \abbr{EC} hit above preset thresholds and $e^+$ \abbr{CC} hit above preset thresholds in the same sector. 
%			\end{enumerate}
%			The lepton trigger ``bit 6" was 100\% efficient (see Sec.~\ref{sec:analysis.trigger.verify}) when the data was cut using all the conditions listed above (1, 2, 3, 4) using an ``OR" flag. This means that a $\gamma p \to p e^+ e^-$ event must satisfy at least one of the listed conditions. The reduction in events when at least one of the conditions was satisfied was 69.91\%. Prior to simulating the trigger, cutting the \abbr{MC} with the listed conditions reduced the event yield by 81.91\%. Simulating the trigger and cutting on the \abbr{MC} events with the listed conditions reduced that event yield to 69.48\%. This indicates that the trigger simulation is properly mimicking the trigger configuration used when data is collected. 
%
%\section{PLUTO++ Event Generator}\label{sec:pluto}
%
%	Pluto~\cite{PLUTO} is a Monte-Carlo event generator designed for the study of hadronic interactions and heavy ion reactions in \abbr{HADES}, \abbr{FAIR} and upcoming \abbr{PANDA} collaborations. The versatility of Pluto enables its use as an event generator for photoproduction in \abbr{CLAS}. For hadronic interactions, Pluto can generate interactions from pion production threshold to intermediate energies of a few~GeV per nucleon. The entire software package is based on ROOT and uses ROOT's embedded C++ interpreter to control the generation of events. Programming event reaction can be set up with a few lines of ROOT macro code without detailed knowledge of programming. Some features in Pluto are, but not limited to;
%	\begin{itemize}
%		\item Ability to generate events in phase space.
%		\item Ability to generate events with a continuous bremsstrahlung photon beam.
%		\item Ability to generate events weighted by a user defined $t$-slope.
%		\item Ability to generate events weighted by a user defined cross-section.
%		\begin{itemize}
%			\item Total cross section can be inputted via functional form or histogram.
%			\item Differential cross sections can be inputted via functional forms or histograms for specific beam energies up to 110 histograms relating to intervals of beam energy.
%		\end{itemize}
%		\item Ability to generate events that decay via already established physics parameters, i.e.~transition form factors.
%		\item Ability to generate events that decay via modified established physics parameters.
%		\item Ability to generate events with multiple production channels, weighted by user inputted cross-section probability.
%		\item Ability to generate events with multiple decay channels, weighted by user inputted branching ratio.
%		\item Ability to perform vertex smearing.
%		\item Ability to create virtual detectors.
%	\end{itemize}
%	
	For the analysis presented in this work, Pluto was used in conjunction with known differential cross sections to verify simulation momentum smearing and tagger resolution, Sec.~\ref{sec:analysis.simsmear.verify}. Pluto was also utilized as a phase space generator in this analysis, to perform a ``tune'' on the kinematic fitter, Sec.~\ref{sec:analysis.fitting}, to calculate the acceptance corrections Sec.~\ref{sec:results.acceptance}, and to calculate the normalization Sec.~\ref{sec:results.normalization}.