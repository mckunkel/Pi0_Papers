\section{\label{sec:analysis.simulation}Monte-Carlo}
	There are certain kinematic regions of \abbr{CLAS} in which physics events are not being recorded properly i.e.~the area dividing each sector in \abbr{CLAS}. Furthermore 
	each sector in \abbr{CLAS} is asymmetric in the acceptance of events due to subsystem inefficiencies such as inoperable Drift Chamber(\abbr{DC}) wires, 
	photomultiplier(\abbr{PMT}) inefficiencies, dead scintillator strips in the Time-of-Flight(\abbr{TOF}) and Start Counter(\abbr{ST}) subsystems. When a triggered event is 
	recorded and reconstructed these asymmetric inefficiencies factors are reflected and must be carefully understood because these factors are properties of the \abbr{CLAS} 
	detector and independent of any physics that occurred. To properly understand the detector effects on the data, \abbr{CLAS} utilizes a \abbr{GEANT3} simulation package 
	know as \abbr{GSIM}. 
	The acceptance for the cross-sections presented in this work was measured using phase-space Monte-Carlo (\abbr{MC}\label{abbr:mc}) simulation, using 
	PLUTO++~\cite{PLUTO} as the generator, for the reaction channels,
	\begin{subequations}
		\begin{align}
			\gamma \p \rightarrow & \ p \pi^{0} \nonumber \\[-0.4em]
			& \ \hookrightarrow p \gamma \gamma \nonumber \\[-0.4em]
			& \qquad \hookrightarrow p e^+ e^- \gamma
			\label{eq:simproduct1}
		\end{align}
		\begin{align}
			\gamma \p \rightarrow & \ p \pi^{0} \nonumber \\[-0.4em]
			& \ \hookrightarrow p e^+ e^- \gamma.
			\label{eq:simproduct2}
		\end{align}
	\end{subequations} 
		A total of events, $N_g$, was generated and this number was weighted by the relative branching ratios found in~\cite{pdg2014} to resemble the conditions of 
		the data. The number of events generated for the reaction channel~\ref{eq:simproduct1} can be found in Table~\ref{tab:simnumspecs} as $N_c$. The number of events 
		generated for the reaction channel~\ref{eq:simproduct2} can be found in Table~\ref{tab:simnumspecs} as $N_d$.
		\begin{table}[h!]
\begin{center}

\caption[Generated Quantities]{\label{tab:simnumspecs}Number of generated events in each decay spectrum}

\begin{tabular}{c|c|c}

%\hline \hline
%
%operation & \multicolumn{3}{c}{Generation} \\
%charge & I & II & III \\
\hline
Quantity & Value$\cdot 10^7$ & Description\\
\hline

$N_g$ & $240$ & Total number of \piz events generated \\
$N_c$ & $237 $ &  Total number of \piz $\rightarrow \gamma \gamma$ events generated\\
$N_d$ & $2.8$ & Total number of \piz $\rightarrow e^+ e^- \gamma$ generated\\
\hline \hline
\end{tabular}

\end{center}
\end{table}
\vspace{20pt}
		After the events are generated, they are inputted into the \abbr{CLAS} simulation chain and then reconstructed with the same program used to reconstruct the data. Once 
		the events are reconstructed, the cuts described in Sec.~\ref{sec:analysis.cuts} are applied. The 
		acceptance $\eta(E_\gamma,\theta^{\pi^0}_{C.M.})$ is then determined by adding the simulations for the conversion and the dalitz, then for photon energy bins of 25 MeV 
		increments and $\Delta\cos\theta^{\pi^0}_{C.M.} = 0.0125$ increments, the ratio of reconstructed events ($N_R$) to generated events ($N_G$) yields,
		\begin{equation}\label{eq:acceptance}
		\eta(E_\gamma,\cos\theta^{\pi^0}_{C.M.}) = \frac{N_R(E_\gamma,\cos\theta^{\pi^0}_{C.M.})}{N_G(E_\gamma,\cos\theta^{\pi^0}_{C.M.})} \ .
		\end{equation}
		The $\Delta\cos\theta^{\pi^0}_{C.M.}$ binning in the acceptance is a factor of 2.4 finer than the smallest $\Delta\cos\theta^{\pi^0}_{C.M.}$ increment used in the 
		cross-section measurement.