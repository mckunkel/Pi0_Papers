\documentclass[%
reprint,
showpacs,
preprintnumbers,
amsmath,amssymb,
aps,
%linenumbers
]{revtex4-1}

\usepackage{graphicx}% Include figure files
\usepackage{dcolumn}% Align table columns on decimal point
\usepackage{bm}% bold math
\usepackage[caption=false]{subfig}
\usepackage{hyperref}% add hypertext capabilities
\usepackage[mathlines]{lineno}% Enable numbering of text and display math
%\linenumbers\relax % Commence numbering lines
\usepackage{placeins} %to Control figure placement with \FloatBarrier
\begin{document}
%%% new commands and macros %%%%%%%%%%%%%%%%%%%%%%%%%%%%%%%%%%%%%%%%%%%
\newlength{\figwidth}
\setlength{\figwidth}{0.9\columnwidth}

\newlength{\qfigheight}
\setlength{\qfigheight}{0.25\textheight}

\newlength{\hfigheight}
\setlength{\hfigheight}{0.5\textheight}

\newcommand{\acro}[1]{#1\@}
\newcommand{\abbr}[1]{\textsc{\texttt{#1}}}
\newcommand{\abbrlc}[2]{\textsc{\texttt{#1}}\texttt{#2}}
\newcommand{\desg}[1]{\texttt{#1}}
\newcommand{\todo}[1]{\textbf{\uppercase{\emph{#1}}}} %\textcolor{Orange}{#1}}}

\def\g12{\emph{g12}}

\def\clas{\abbr{CLAS }}

\def\figures{figures}
\def\tablepath{../../}
\newcommand{\bank}[4]{$\mathtt{#1}^{#2}_{#3}\lbrack\mathtt{#4}\rbrack$}

\def\ith{$i$\textsuperscript{th}}



%\def\coloronline{(Color online.)\ }
\def\coloronline{}

%%% particles
\def\p{\mathrm{p}}
\def\n{\mathrm{n}}
\def\Kp{\mathrm{K}^{+}}
\def\Km{\mathrm{K}^{-}}
\def\K0{\mathrm{K}^{0}}
\def\Y{\mathrm{Y}}
\def\epos{\mathrm{e}^{+}}
\def\eneg{\mathrm{e}^{-}}
\def\gamstar{\mathrm{$\gamma$}^{*}}
\def\piup{$\pi$}
\def\gammaup{$\gamma$}
\def\um{{\text{$\mu$}}m}

\newcommand{\bra}[1]{\left<#1\right|}
\newcommand{\ket}[1]{\left|#1\right>}
\newcommand{\braket}[2]{\left<#1\middle|#2\right>}
\def\piz{$\mathrm{\pi^{0}}\ $}
\def\epem{$e^+e^-\ $}
\preprint{APS/123-QED}

\title{Photoproduction of the $\pi^0$ meson off protons from 2.775 to 5.45~GeV}
\author{M. C. Kunkel}
\altaffiliation[Now at ]{Forschungszentrum J\"ulich}%Lines break automatically or can be forced with \\
\email{m.kunkel@fz-juelich.de}
\author{M. J. Amaryan}%
\affiliation{%
	Old Dominion University, VA (U.S.A.)
}
\author{I. Strakovsky}%
\affiliation{%
	The George Washington University, Washington, DC (U.S.A.) 
	\\
}
\collaboration{and the CLAS Collaboration }%\noaffiliation
\date{\today}% It is always \today, today,
             %  but any date may be explicitly specified

\begin{abstract}
Exclusive neutral pion photoproduction ($\gamma p \rightarrow p \pi^0$) was measured in the CLAS detector at the Thomas Jefferson National Facility. The experiment employed a 1.1-5.5~GeV bremsstrahlung photon beam from 5.6~GeV electron beam created in the Continuous Electron Beam Accelerator Facility (CEBAF). The photon beam energy was impinged on a liquid hydrogen target. The neutral pions were detected via external conversion, $\pi^0 \rightarrow \gamma \gamma \rightarrow e^+e^-\gamma$, and subsequent Dalitz decay, $\pi^0 \rightarrow \gamma^* \gamma \rightarrow e^+e^-\gamma$. Measured differential cross-sections, $\frac{d\sigma}{dt}$ and $\frac{d\sigma}{d\cos \theta}$ are compared with the constituent counting rule, Regge and handbag theoretical calculations. The results for the constituent counting rule agree well with the data. The Regge theoretical calculations underestimate the differential cross sections between 3.9 and 4.6~GeV, but agree with data at photon energies 4.6-5.4~GeV. The handbag theoretical calculation significantly underestimates the data at center of mass energies, s  $\sim$ 11~GeV.  
\end{abstract}
%
%\pacs{13.20.Cz, 13.25.Cq, 13.30.Ce,  13.60.Le, 14.40.0n}% PACS, the Physics and Astronomy
                             % Classification Scheme.
%\keywords{Suggested keywords}%Use showkeys class option if keyword
                              %display desired
\maketitle
%
%\tableofcontents
\section{Introduction}\label{sec:intro}
	In hadron physics, photoproduction of single pion is essential to understand the photon-nucleon vertex. At low energies, the photon-nucleon coupling establishes excited nucleon resonances which has been at the forefront of physics ''missing resonances'' search. At high energies single pion photoproduction can be used to test predictions of Regge theory, in which recent calculations~\cite{JPAC} have shown to describe the presented data well. Furthermore, these measurements have shown that the differential cross section for single pion photoproduction at fixed c.m. angles, $\theta_{c.m.}$, where the mandelstam variables s, t, and u are large, seem to scale as $\frac{d\sigma}{dt} \sim s^{2-n}f(\theta_{c.m.})$, where $s$ and $t$ are the Mandelstam variables and $n$ is the total number of interacting elementary fields in the initial and final state of the reaction. This is predicted by the constituent counting rule~\cite{scaling1,scaling2} and exclusive measurements in $pp$ and  $\bar{p}p$ elastic scattering~\cite{scalingexp5, scalingexp7}, meson-baryon $M p$ reactions~\cite{scalingexp7}, and photoproduction $\gamma N$~\cite{scalingexp2, scalingexp3, scalingexp4, scalingexp6, scalingexp8, scalingexp9, scalingexp10, scalingexp11} agree well with this rule.
	
	This analysis note details the CLAS g12 data set, the extraction of the \pizT signal from the data, the Monte-Carlo techniques utilized for acceptance correction and the track efficiency calculation used for correcting the data. Also to be shown is the differential cross-sections through the entire beam energy range of the g12 experiment, a comparison of the differential cross-section with existing world data. 
	
	This analysis note details the analysis techniques and corrections not already discussed and approved in the g12 analysis note procedure document~\cite{g12note}. All relevant procedures described in~\cite{g12note} have been applied to this analysis. See checklist.
	
	%of $70^{\circ}$, $90^{\circ}$ and $110^{\circ}$

\section{\label{sec:level1}Introduction}
Write the experiment here



\section{\label{sec:level1}Event Selection}
Write 
\section{\label{sec:level1}Particle Identification}
Lepton identification was based on conservation of mass. Once the data is skimmed according to Table~\ref{tab:skim.requirements}, all particles that were $\pi^+$, $\pi^-$, unknown with $q^+$ or unknown with $q^-$ were tentatively assigned to be electrons or positrons based on their charge. This meant that the mass term of the particle's 4-vector was set to be the mass of an electron instead of that of a pion. This technique works because the mass of the \piz (0.135~GeV) is less than the mass of $\pi^+$ or $\pi^-$ (0.139~GeV) and by laws of conservation of energy-momentum, a lighter particle cannot decay into heavier particle's.
\section{\label{sec:level1}Monte-Carlo}
Write 
\section{\label{sec:level1}Systematic Uncertainties}
 Systematic errors are caused the controls of the experiment, such as flux, simulation, density and length of the $\ell H_2$ target and also systematic errors are caused by various analytical tools used, such as the kinematic fitter.
 \subsection{Branching Ratio Systematic Uncertainty}
 The branching ratios for the two topologies used to measure the cross-section were obtained from~\cite{pdg2014} and are listed again in Table~\ref{tab:brspecs} with their associated errors. Uncorrelated quantities that are summed as,
 \begin{align}
 	f = \sum_{i = 1}^{M}a_iP_i  
 \end{align}
 have errors as
 \begin{align}
 	\sigma_f = \sqrt{\sum_{i = 1}^{M}\left(a_i\sigma_i\right)^2}.  
 \end{align}
 Therefore
 \begin{align}
 	\frac{\Gamma}{\Gamma_{tot}} &  = \frac{\Gamma_{\pi^{0}\rightarrow e^{+}e^{-}\gamma}}{\Gamma_{tot}} + \frac{\Gamma_{\pi^{0}\rightarrow \gamma \gamma \to e^{+}e^{-}\gamma}}{\Gamma_{tot}}  \\ & = \frac{\Gamma_{\pi^{0}\rightarrow e^{+}e^{-}\gamma}}{\Gamma_{tot}} + \frac{\Gamma_{\pi^{0}\rightarrow \gamma \gamma}P(\gamma \to  e^{+}e^{-})}{\Gamma_{tot}} \ ,
 \end{align}
 where $P(\gamma \to  e^{+}e^{-})$ is the probability of photon conversion into $e^+e^-$. To measure $P(\gamma \to  e^{+}e^{-})$, the acceptance for conversion ($P(\gamma \to  e^{+}e^{-})\cdot\eta_{e^+e^-}$) is divided by the acceptance for Dalitz ($\eta_{e^+e^-}$). The conversion probability depends on incident photon energy. A maximum probability of 8\%  per-photon was measured.
 	Therefore,
 	\begin{align}
 		\frac{\Gamma}{\Gamma_{tot}} = \frac{\Gamma_{\pi^{0}\rightarrow e^{+}e^{-}\gamma}}{\Gamma_{tot}} + \frac{\Gamma_{\pi^{0}\rightarrow \gamma \gamma}P(\gamma \to  e^{+}e^{-})}{\Gamma_{tot}} = 0.09 \ ,
 	\end{align}
 	and has error
 	\begin{align}
 		\sigma_f = \sqrt{\left(\frac{1}{\Gamma_{tot}}\right)^2(\sigma^2_{\pi^{0}\rightarrow e^{+}e^{-}\gamma} + \sigma^2_{\pi^{0}\rightarrow \gamma \gamma})  } = 0.0037.  
 	\end{align}
 	The energy and $\cos \theta$ dependence of the conversion is accounted for in the acceptance, which is $E_\gamma$ and $\cos \theta$ bin-dependent. 
 	\input{tables/branchingratios.tex} 	
 \subsection{Cut Based Systematic Uncertainty}
	 The procedure to determine the systematic uncertainty of the cuts placed on the various kinematic fits was first to calculate an acceptance with a different cut, then to calculate a new total cross-section measurement applying the different cut to the data. The total cross-section was computed at various photon beam energies. Lets denote the original measured total cross-section as $\Xi_1$ and the new total cross-section determined by the new cut as $\Xi_n$, then the systematic error was calculated as.
 	
 	\begin{align}
 		\sigma_{cut} = \frac{\left| \Xi_1 - \Xi_n \right|}{\Xi_1}
 	\end{align}
 	
 	Some systematic uncertainty depended on the photon energy. All cut based systematics were performed individually, meaning when a cut was changed, the remaining cuts retained their original value, see Table~\ref{tab:cutsystematics} for the values of the cuts that were changed to calculate the systematic error.
 	%This new cross-section was then compared to the cross-section using the base cuts to determine if a dependence on $\cos\theta^{\pi^0}_{C.M.})$ was present. All systematics of the cut based did not show a $\cos\theta^{\pi^0}_{C.M.})$ dependency for specific beam energies.
 	\begin{table}[h!]
\begin{center}


\caption[Variance of Data Cut Systematics]{\label{tab:cutsystematics}Different Cuts to analyze systematics \vspace{0.75mm}}

%\begin{tabular}{c|c|c|c}
\begin{tabular}{p{2.75cm} | p{1.25cm}|p{1.25cm} | p{3.4cm}}
\hline
Cut & Original & Adjusted & Uncertainty \\
\hline
2-C Fit  & 1\% & 10\% & $0.0219$\\
1-C Fit  & 1\% & 10\% & $ 0.00216 + 0.01083E_{\gamma}$ \\
4-C Fit  & 1\% & 10\% & $0.00031$\\
Missing Energy  & 75~MeV & 100~MeV & $0.02781$\\
\hline \hline
\end{tabular}


\end{center}
\end{table}
\vspace{20pt}
% 	
% 	\begin{figure}[h!]\begin{center}
% 			\includegraphics[width=1.2 \figwidth,height=\hfigheight]{\figures/analysis/All_Cut_Systematic.pdf}
% 			\caption[Plot showing the contribution of the data cut systematic error and the incoming beam dependence of the error]{\label{fig:sys_cut_error} Plot showing the contribution of the data cut systematic error and the incoming beam dependence of the error.}
% 		\end{center}\end{figure}
% 	 	
\subsection{Photon Flux Systematic Uncertainty}
 		The photon flux calculation should be consistent throughout the experiment. If the flux measurement is not consistent due to corrections made with the live-time, beam corrections or fractional difference in the reported current to the actual current during the photon flux normalization run then a systematic uncertainty would be produced. It was deduced that for the entire \g12 run, the lower limit for the flux normalization uncertainty was 6\%. 
 		% Table~\ref{tab:flux_sys} lists the run groups used for this study.
 		%\input{tables/sys_flux_runs.tex}
% 		
% 		To study this effect we divided the g12 run period into four sequential groups. The procedure to determine the systematic error, $\sigma$, of the flux is to calculate the accepted and flux corrected yield, $\Upsilon^c$, for each run group and compare $\Upsilon^c$ to the average accepted and flux corrected yield of all 4 run groups, $\mu^c$. After $\sigma$ is calculated, it was normalized to $N \mu^c$ as to represent the error as a percentage, which later is added in quadrature and  multiplied by the measured cross section to determine the appropriate error. 
% 		\begin{align}
% 			\sigma_{group} = \sqrt{\sum_{i=1}^{N = 4}\left(\Upsilon_i^c - \mu^c\right)},
% 		\end{align}
% 		where
% 		\begin{align}
% 			\mu^c = \frac{1}{N}\sum_{i=1}^{N=4}\Upsilon_i^c
% 		\end{align}
% 		\begin{align}
% 			\sigma_{group}^{normalized} = \frac{\sigma_{group}}{N\mu^c}
% 		\end{align}
 		
% 		\begin{figure}[h!]\begin{center}
% 				\includegraphics[width=1.2 \figwidth,height=\hfigheight]{\figures/analysis/Beam_Fluxsystematic_Error.pdf}
% 				\caption[Plot showing the contribution of the flux systematic error and the incoming beam dependence of the error]{\label{fig:sys_flux_error} Plot showing the contribution of the flux systematic error and the incoming beam dependence of the error.}
% 			\end{center}\end{figure} 	 	
\subsection{Detector Efficiency Systematic Uncertainty}
 	Each sector in \abbr{CLAS} can be treated as an individual detector, with its own efficiency and resolution. A systematic uncertainty could arise if one or more of the sectors is not simulated properly. It was determined that this systematic uncertainty was incident beam energy dependent.
% 		The procedure to determine the systematic error, $\sigma$, of the sector is to calculate the accepted corrected yield, $\Upsilon^c$, for each sector and compare $\Upsilon^c$ to the average accepted corrected yield of all 6 sectors, $\mu^c$. After $\sigma$ is calculated, it was normalized to $N \mu^c$ as to represent the error as a percentage, which later is multiplied by the measured cross section to determine the appropriate error. 
% 		\begin{align}
% 			\sigma_{sector} = \sqrt{\sum_{i=1}^{N = 6}\left(\Upsilon_i^c - \mu^c\right)},
% 		\end{align}
% 		where
% 		\begin{align}
% 			\mu^c = \frac{1}{N}\sum_{i=1}^{N=6}\Upsilon_i^c
% 		\end{align}
% 		\begin{align}
% 			\sigma_{sector}^{normalized} = \frac{\sigma_{sector}}{N\mu^c}
% 		\end{align}
% 		This calculation was performed for various bins of incoming beam energy to determine the beam energy dependence.% (see Fig.~\ref{fig:sys_sec_error}).
 			
% 			\begin{figure}[h!]\begin{center}
% 					\includegraphics[width= \figwidth,height=0.75\hfigheight]{\figures/analysis/Beam_systematic_Error.pdf}
% 					\caption[The sector systematic uncertainty as a function of the incoming photon energy]{\label{fig:sys_sec_error}The sector systematic uncertainty as a function of the incoming photon energy.}
% 				\end{center}\end{figure}
 				%The sector systematic uncertainty is consistent with the extracted sector systematic uncertainty from the g11 data set~\cite{williams}(seeFig.~\ref{fig:sys_sec_error.compare}).
% 				\begin{figure}[h!]\begin{center}
% 						\includegraphics[width= \figwidth,height=0.75\hfigheight]{\figures/analysis/S_systematic_Error.pdf}
% 						\caption[Comparison of sector systematic uncertainty to g11 measurement]{\label{fig:sys_sec_error.compare}Comparison of sector systematic uncertainty to g11 measurement.}
% 					\end{center}\end{figure}

 					 	 	
\subsection{$z$-vertex Cut Systematic Uncertainty}
 	The systematic uncertainty of the $z$-vertex cut was analyzed by varying the initial vertex cut from $-110 \le z \le -70$ to $-109 \le z \le -71$ for both data and \abbr{MC}. Afterward the procedure for determining the systematic was identical to the method used to determine the ``Cut Based Systematic Uncertainty". The systematic uncertainty from varying the $z$ was 0.0041. 	 	
\subsection{Target Systematic Uncertainty}
 	Since the systematic on the density is 0.02\% the maximum systematic on the target is due to uncertainty in the length on the target which is 40~cm $\pm$ 0.2~cm. A total systematic on the target was assign to be 0.5\%.  	 	
\subsection{Total Systematic Uncertainty}
 	The total systematic uncertainty along with a list of the individual systematics is presented in this subsection. The calculation of the total systematic error is 
 	\begin{align}
 		\sigma^{sys}_{tot} = \sqrt{\sum_{i=1}^{M}\sigma_i^2}
 	\end{align}
 	\begin{table}[h!]
\begin{center}


\caption[Systematics]{\label{tab:systematics}Relative Systematic Uncertainty used in the $\frac{d\sigma}{d\cos\theta^{\pi^0}_{C.M.} d\phi}$ measurements. \vspace{0.75mm}}

%\begin{tabular}{c|c}
\begin{tabular}{p{5.25cm} | p{5.35cm}}
\hline
Relative Systematic & Error \\
\hline
Particle Efficiency (total) & $0.005$ \\
Sector  & $ 0.0361 + 0.0065E_{\gamma}$ \\
Flux  & $ 0.057$ \\
Missing Energy Cut  & $0.02781$ \\
2-C Fit Pull Probability & $0.0219$ \\
1-C Fit Pull Probability  & $ 0.00216 + 0.01083E_{\gamma}$ \\
4-C Fit Pull Probability  & $0.00031$ \\ 
Target  & $0.005$ \\
Branching Ratio  & $0.0037$ \\
Fiducial Cut & $0.024$ \\
$z$-vertex Cut & $0.0041$ \\
%Total & $(0.0032 +0.00051E_{\gamma} +0.000184E_{\gamma}^2)^{\frac{1}{2}}$ \\
Total & $\sqrt{(6.5 +0.52E_{\gamma} +0.16E_{\gamma}^2)\cdot10^{-3}}$ \\
\hline \hline
\end{tabular}


\end{center}
\end{table}
\vspace{20pt}
 	For the data presented in this manuscript the lowest systematic uncertainty, at $E_\gamma = 2.775$~GeV, is 9.72\%, while the highest systamtic uncertainty, at $E_\gamma = 5.425$~GeV, is 11.9\%.
% 					Figure~\ref{fig:results.syserr} is a pictorial version of Table~\ref{tab:systematics}.
% 					\begin{figure}[h!]\begin{center}
% 							\includegraphics[width=1.1 \figwidth,height=\hfigheight]{\figures/SYSTEMATICS/All_Systematics.pdf}
% 							\caption[The contribution of all systematic uncertainties]{\label{fig:results.syserr}The contribution of all systematic uncertainties.}
% 						\end{center}\end{figure}  	 	

 				
%\section{\label{sec:results}Cross-Sections}
Write 
\section{Comparison with Theoretical Models}\label{sec:compare}
There are several models that attempt to describe  $\pi^0$ photoproduction in the low beam energy resonance region, while in the high beam energy regime there exists limited amount of theory. Described below are two theories. 
\subsubsection{HandBag Model}
The production of the \piz meson in photon-proton reactions, for incoming photon beam energies greater than 2.8~GeV, can considered to be a hard exclusive reaction. One approach to study the \piz photoproduction, is use the handbag model. In the handbag approach, the reaction is factorized into two parts. The first part is when one quark from the incoming and one from the outgoing nucleon participate in the hard sub-process. This hard sub-process is achieved when the incident photon excites a quark, since quarks are bound quantum particles, the excited quark produces a jet of quarks that form the meson and then de-excites back into the nucleon. This is calculable using pQCD. The second part ,the soft part, consists of all the other quarks that are spectators and can be described in terms of GPDs~\cite{key1, key2,Rad1996, Diehl}. The handbag mechanism is applicable when the Mandelstam variables, $s$, $t$, $u$, are large as compared to a hadronic scale of order 1 GeV . In Ref.~\cite{Huang2000} a model, derived from the handbag approach, has been applied to predict angular dependence of scaled photoproduction cross section of \piz and is illustrated in Fig.~\ref{fig:pi0_handbag}. The handbag model calculations by Kroll \textit{et al.}~\cite{Huang2000} does not agree with the measurement obtained by \g12.
\begin{figure}[h]
	\includegraphics[width=200 pt]{\figures/analysis/DSG/kroll-eps-converted-to.pdf}
	\caption{Comparison of the $\pi^0$ differential cross section  photoproduction data to \abbr{GDP} handbag model. Experimental data at $s$ = 11.08~GeV$^2$ are from the current (red filled circles). The theoretical prediction at $s$ = 10~GeV$^2$ by~\protect\cite{Huang2000} is given by blue solid line. } \label{fig:pi0_handbag}
\end{figure}
%\subsection{Constituent Counting Rule}
	The constituent counting rule (CCR) predicts the energy dependence of the differential cross-section at fixed center-of-mass angles for an exclusive two-body reactions. It validity is at high beam energies and large momentum transfer and the framework is similar to that of the handbag approach, in which the theory relies on the factorization of the exclusive process into a hard scattering amplitude and a soft quark amplitude inside the hadron. The prediction of CCR is:
	\begin{equation}
		\frac{d\sigma}{dt} \sim s^{2-n}f(\theta_{c.m.}) \label{CCR}
	\end{equation}
	where $s$ and $t$ are the Mandelstam variables, $n$ is the total number of interacting elementary fields in the initial and final state of the reaction and $f(\theta_{c.m.})$ depends on the dynamics of the process. Many exclusive measurements in $pp$ and  $\bar{p}p$ elastic scattering~\cite{scalingexp5, scalingexp7}, meson-baryon $M p$ reactions~\cite{scalingexp7}, and photoproduction $\gamma N$~\cite{scalingexp2, scalingexp3, scalingexp4, scalingexp6, scalingexp8, scalingexp9, scalingexp10, scalingexp11} agree well with this rule. For \piz photoproduction reactions CCR predicts that the differential cross-section $\frac{d\sigma}{dt}$ should scale as $s^{-7}$, where -7 was calculated from 4 elementary fields in the initial state, 1 for the photon, 3 for the number of quarks in a proton, and 5 elementary fields in the final state, 3 quarks from the proton and  2 quarks from the \piz, 2-9 =-7. A comparison of this previous data along with the \g12 measurements can be seen in figure~\ref{fig:pi0_scaling}, at high energies and large angles the results are consistent with the $s^{−7}$ scaling expected from the quark counting rule. 
	\begin{figure}[h]
		\centerline{\includegraphics[width=250 pt, height=200 pt]{\figures/ANALYSIS/pi0_scaling.pdf}}
		\caption{The differential cross section for the $\gamma p \to p \pi^0$ reaction at $\theta_{c.m.}$ = $50^{\circ}$, $70^{\circ}$, $90^{\circ}$, $110^{\circ}$, as a function of s (center of mass energy squared). Experimental data are from the current measurement (red filled circles), CLAS~\protect\cite{Dugger07,Dugger13} (blue circles), MAMI~\protect\cite{beck} (magenta circles), old measurements~\protect\cite{Joos} (black open circle plus). The dash dotted line is a result of the fit performed at $\theta$ = $90^{\circ}$ with power function $\sim s^{−n}$ leading to n = $6.89 \pm 0.26$.}
		\label{fig:pi0_scaling}
	\end{figure}
	\FloatBarrier
%\section{\label{sec:summary}Summary}
\nocite{*}
\bibliographystyle{aipnum-cp}%
\bibliography{PI0}%
\end{document}