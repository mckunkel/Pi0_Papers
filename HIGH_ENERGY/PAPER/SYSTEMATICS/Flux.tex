\subsection{Photon Flux Systematic Uncertainty}
 		The photon flux calculation should be consistent throughout the experiment. If the flux measurement is not consistent due to corrections made with the live-time, beam corrections or fractional difference in the reported current to the actual current during the photon flux normalization run then a systematic uncertainty would be produced. It was deduced that for the entire \g12 run, the lower limit for the flux normalization uncertainty was 6\%. 
 		% Table~\ref{tab:flux_sys} lists the run groups used for this study.
 		%\begin{table}[h!]
\begin{minipage}{\textwidth}
\begin{center}


\caption[Run Groups Used to Determine Photon Flux Systematic Error ]{\label{tab:flux_sys}List of run groups used to determine photon flux systematic error. \vspace{0.75mm}}

\begin{tabular}{c|c|c}

\hline
Run Group & Range & Total Runs \\
\hline
1 & 56605-56798 & 116 \\
2 & 56799-56980 & 116 \\
3 & 56992-57173 & 116 \\
4 & 57174-57317 & 115 \\
\hline \hline
\end{tabular}


\end{center}
\end{minipage}
\end{table}
\vspace{20pt}
% 		
% 		To study this effect we divided the g12 run period into four sequential groups. The procedure to determine the systematic error, $\sigma$, of the flux is to calculate the accepted and flux corrected yield, $\Upsilon^c$, for each run group and compare $\Upsilon^c$ to the average accepted and flux corrected yield of all 4 run groups, $\mu^c$. After $\sigma$ is calculated, it was normalized to $N \mu^c$ as to represent the error as a percentage, which later is added in quadrature and  multiplied by the measured cross section to determine the appropriate error. 
% 		\begin{align}
% 			\sigma_{group} = \sqrt{\sum_{i=1}^{N = 4}\left(\Upsilon_i^c - \mu^c\right)},
% 		\end{align}
% 		where
% 		\begin{align}
% 			\mu^c = \frac{1}{N}\sum_{i=1}^{N=4}\Upsilon_i^c
% 		\end{align}
% 		\begin{align}
% 			\sigma_{group}^{normalized} = \frac{\sigma_{group}}{N\mu^c}
% 		\end{align}
 		
% 		\begin{figure}[h!]\begin{center}
% 				\includegraphics[width=1.2 \figwidth,height=\hfigheight]{\figures/analysis/Beam_Fluxsystematic_Error.pdf}
% 				\caption[Plot showing the contribution of the flux systematic error and the incoming beam dependence of the error]{\label{fig:sys_flux_error} Plot showing the contribution of the flux systematic error and the incoming beam dependence of the error.}
% 			\end{center}\end{figure}