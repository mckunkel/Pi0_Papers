 \subsection{Cut Based Systematic Uncertainty}
	 The procedure to determine the systematic uncertainty of the cuts placed on the various kinematic fits was first to calculate an acceptance with a different cut, then to calculate a new total cross-section measurement applying the different cut to the data. The total cross-section was computed at various photon beam energies. Lets denote the original measured total cross-section as $\Xi_1$ and the new total cross-section determined by the new cut as $\Xi_n$, then the systematic error was calculated as.
 	
 	\begin{align}
 		\sigma_{cut} = \frac{\left| \Xi_1 - \Xi_n \right|}{\Xi_1}
 	\end{align}
 	
 	Some systematic uncertainty depended on the photon energy. All cut based systematics were performed individually, meaning when a cut was changed, the remaining cuts retained their original value, see Table~\ref{tab:cutsystematics} for the values of the cuts that were changed to calculate the systematic error.
 	%This new cross-section was then compared to the cross-section using the base cuts to determine if a dependence on $\cos\theta^{\pi^0}_{C.M.})$ was present. All systematics of the cut based did not show a $\cos\theta^{\pi^0}_{C.M.})$ dependency for specific beam energies.
 	\begin{table}[h!]
\begin{center}


\caption[Variance of Data Cut Systematics]{\label{tab:cutsystematics}Different Cuts to analyze systematics \vspace{0.75mm}}

%\begin{tabular}{c|c|c|c}
\begin{tabular}{p{2.75cm} | p{1.25cm}|p{1.25cm} | p{3.4cm}}
\hline
Cut & Original & Adjusted & Uncertainty \\
\hline
2-C Fit  & 1\% & 10\% & $0.0219$\\
1-C Fit  & 1\% & 10\% & $ 0.00216 + 0.01083E_{\gamma}$ \\
4-C Fit  & 1\% & 10\% & $0.00031$\\
Missing Energy  & 75~MeV & 100~MeV & $0.02781$\\
\hline \hline
\end{tabular}


\end{center}
\end{table}
\vspace{20pt}
% 	
% 	\begin{figure}[h!]\begin{center}
% 			\includegraphics[width=1.2 \figwidth,height=\hfigheight]{\figures/analysis/All_Cut_Systematic.pdf}
% 			\caption[Plot showing the contribution of the data cut systematic error and the incoming beam dependence of the error]{\label{fig:sys_cut_error} Plot showing the contribution of the data cut systematic error and the incoming beam dependence of the error.}
% 		\end{center}\end{figure}
%