\section{\label{sec:analysis.pid}Particle Identification}
Lepton identification was based on conservation of mass. Once the data is skimmed according to Table~\ref{tab:skim.requirements}, all particles that were $\pi^+$, $\pi^-$, unknown with $q^+$ or unknown with $q^-$ were tentatively assigned to be electrons or positrons based on their charge. This meant that the mass term of the particle's 4-vector was set to be the mass of an electron instead of that of a pion. This technique works because the mass of the \piz (0.135~GeV) is less than the mass of $\pi^+$ or $\pi^-$ (0.139~GeV) and by laws of conservation of energy-momentum, a lighter particle cannot decay into heavier particle's.
\subsection{Kinematic Cuts}
First it should be noted that for the /g12 experiment, there was a two-prong trigger for events in which the photon beam energy was greater then 3.6~GeV, while for the entire data taking process there was a ``lepton'' trigger configuration. Therefore to measure the differential cross-section at photon beam energies less than 3.6~GeV this ``lepton'' trigger information was employed. 
Once particle section was achieved, it was necessary to reduce the background of the exclusive $\gamma p \to p \pi^{+}\pi^{-}$ reaction. For events of photon beam energy less than 3.6~GeV, a \abbr{CC} and \abbr{EC} ``hit'' must have been recorded for each charge track that was not the proton. For all events 3 kinematic fits were performed, a 1-C ( $\gamma p \to p e^{+}e^{-}(\gamma)$) to identify the missing photon in the reaction, a 4-C ($\gamma p \to p \pi^{+}\pi^{-}$) as a discriminator and a 2-C ($\gamma p \to p \pi^0 \to pe^{+}e^{-}(\gamma)$) to identify the reaction. After the kinematic fit, a 1\% confidence level cut was placed on the 1-C fit. The missing energy of the $\gamma p \to p e^{+}e^{-}$ spectrum versus the missing mass of $\gamma p \to p X$ was analyzed and shown that a 75~MeV cut on the missing energy was suitable to suppress the $\gamma p \to p \pi^{+}\pi^{-}$ reaction, see figure~\ref{fig:Mxp}. After the missing energy cut, the signal to background ratio was $\sim 99.7\%$, see right figure~\ref{fig:Mxp}, the other 4-C and 2-C fits were then used with a 1\% confidence on each to suppress the background to $\sim 99.9\%$.
 \begin{figure}[h!]\begin{center}
 		\subfloat[Probability of Photon Conversion vs. $\cos\theta$][]{ %Feynman diagram of \piz two photon decay
        \includegraphics[width=0.8\columnwidth,height=0.75 \hfigheight]{\figures/analysis/KineFitter/DATA/mm2P_vs_mEPEpEm.pdf}\label{fig:kinefit.mm2p.mE.data}
 		}\\
 		\subfloat[Probability of Photon Conversion vs. $\cos\theta$][]{ %Feynman diagram of \piz Dalitz decay
        \includegraphics[width=\figwidth,height= 0.75 \hfigheight]{\figures/analysis/KineFitter/DATA/hdataLEP_MOR_pi0_FINAL_PLOTS.pdf}
 		}
		\caption{Left: $M_x^2 (p)$ vs. $M_E(pe^+e^-)$. The horizontal red dashed-dotted line depicts the 75 MeV cut used in this analysis. The vertical red dashed-dotted line depicts the boundary of single $\pi^0$ to $\pi^{+}\pi^{-}$ production. Right: Final $M_x^2(p)$ data used in analysis. The horizontal red dashed-dotted line depicts the threshold of $\pi^{+}\pi^{-}$ production.}{}
		\label{fig:Mxp}
 	\end{center}\end{figure}
 	