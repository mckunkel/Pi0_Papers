\subsection{Lepton Triggering and Neutral Triggering}\label{sec.data.trig.lepton}
		In \g12, since the \abbr{CC} was filled with gas, it was possible to include the \abbr{CC} as a component of the trigger. 
		There were three trigger ``bits'' used for lepton identification in \g12 as listed in Table~\ref{tab:data.trig.conf.2}. Each ``bit'' used a (\abbr{EC}$\cdot$\abbr{CC}) configuration to identify leptons. The (\abbr{EC}$\cdot$\abbr{CC}) configuration required a coincidence between the electromagnetic calorimeter and the Cherenkov subsystems. This coincidence was established by using the voltage sum of the \abbr{CC} for a sector and the voltage sum of the \abbr{EC} for the same sector and comparing each sum to a preset threshold described in Table~\ref{tab:data.ecccthresh}. The \abbr{EC} voltage sum threshold comparison is done on both the \abbr{EC}$_\mathrm{inner}$ and \abbr{EC}$_{\mathrm{total}}$ which are the \abbr{EC} voltage signals for the energy deposited in the inner layer and in all layers. The labels of photon or electron specified in Table~\ref{tab:data.ecccthresh} are not actual photons or electrons, but were considered a first-order approximation for detection. The particle identification is done at the analysis level. The method for determining the (\abbr{EC}$\cdot$\abbr{CC}) does not allow for multiple lepton triggering in the same sector. Determining multiple leptons in the same sector is done at the analysis level. 
		
		The ``bit 6'' trigger configuration, (\abbr{ST}$\cdot$\abbr{TOF})$\cdot$(\abbr{EC}$\cdot$\abbr{CC}) had a \abbr{ST} and \abbr{TOF} coincidence. This requirement was to establish a track to have coincidence in one sector between any one of the four start counter paddles of that sector, and any one of the 57 time-of-flight paddles in the same sector. The (\abbr{ST}$\cdot$\abbr{TOF}) configuration of ``bit 6'' did not have to be in the same sector as the (\abbr{EC}$\cdot$\abbr{CC}) configuration of ``bit 6''. The ``bit 11'' trigger configuration, (\abbr{EC}$\cdot$\abbr{CC})$\times$2 requires two coincidences between the electromagnetic calorimeter and the Cherenkov subsystems described above, in two different sectors. 
		
		The ``bit 5'' trigger configuration was also established as a lepton trigger. It required \abbr{EC} hits in two sectors. The ``bit 5'' trigger configuration was also established to analyze physics involving two or more neutral particles accompanied with a charged track, such as exclusive \piz production in which the \piz decays via 2 photons. The method for ``bit 5'' voltage sum comparison is identical to the \abbr{EC} voltage sum of ``bit 6'' and ``bit 11''
		
		It should be noted that none of the lepton triggers required a \abbr{MOR} signal, allowing for physics involving leptons to be measured starting from \g12's lowest tagger detection value of 1.142~GeV.
		
%		\begin{table}
\begin{minipage}{\textwidth}
\begin{center}
\begin{singlespacing}

\caption[Trigger Configuration 1]{\label{tab:data.trig.conf.1}Trigger configuration for \g12 runs from 56363 to 56594 and 56608 to 56647. (\abbr{ST}$\cdot$\abbr{TOF})$_{i}$ indicates a trigger-level track defined as a coincidence between a start counter and time-of-flight hit in the \ith\ sector. \abbr{MORA} and \abbr{MORB} represent coincidences with tagger hits within a certain energy range as specified in Table~\ref{tab:data.trig.mor}.}

\begin{tabular}{cccc}

\hline

\multicolumn{4}{c}{\g12 runs 56363--56594, 56608--56647} \\

\hline

bit & definition & L2 multiplicity & prescale \\

\hline

1 & \abbr{MORA}$\cdot$(\abbr{ST}$\cdot$\abbr{TOF})$_{1}\cdot$(\abbr{ST}$\cdot$\abbr{TOF})$_{i\neq 1}$ & -- & 1 \\
2 & \abbr{MORA}$\cdot$(\abbr{ST}$\cdot$\abbr{TOF})$_{2}\cdot$(\abbr{ST}$\cdot$\abbr{TOF})$_{i\neq 2}$ & -- & 1 \\
3 & \abbr{MORA}$\cdot$(\abbr{ST}$\cdot$\abbr{TOF})$_{3}\cdot$(\abbr{ST}$\cdot$\abbr{TOF})$_{i\neq 3}$ & -- & 1 \\
4 & \abbr{MORA}$\cdot$(\abbr{ST}$\cdot$\abbr{TOF})$_{4}\cdot$(\abbr{ST}$\cdot$\abbr{TOF})$_{i\neq 4}$ & -- & 1 \\
5 & \abbr{MORA}$\cdot$(\abbr{ST}$\cdot$\abbr{TOF})$_{5}\cdot$(\abbr{ST}$\cdot$\abbr{TOF})$_{i\neq 5}$ & -- & 1 \\
6 & \abbr{MORA}$\cdot$(\abbr{ST}$\cdot$\abbr{TOF})$_{6}\cdot$(\abbr{ST}$\cdot$\abbr{TOF})$_{i\neq 6}$ & -- & 1 \\
7 & \abbr{ST}$\cdot$\abbr{TOF} & -- & 1 \\
8 & \abbr{MORA}$\cdot$(\abbr{ST}$\cdot$\abbr{TOF})$\times$2 & -- & 1 \\
11\footnote{bit 11 and \abbr{MORB} were included in the trigger starting with run 56519.} & \abbr{MORB}$\cdot$(\abbr{ST}$\cdot$\abbr{TOF})$\times$2 & -- & 1 \\
12 & (\abbr{ST}$\cdot$\abbr{TOF})$\times$3 & -- & 1 \\

\hline \hline

\end{tabular}

\end{singlespacing}
\end{center}
\end{minipage}
\end{table}
\vspace{20pt}
 % label: tab:data.trig.conf.1
%		
		\begin{table}
\begin{minipage}{\textwidth}
\begin{center}
\begin{singlespacing}

\caption[Trigger Configuration 2]{\label{tab:data.trig.conf.2}Trigger configuration for \g12 runs from 56595 to 56607 and 56648 to 57323. \vspace{0.75mm}}

\begin{tabular}{cccc}

\hline

\multicolumn{4}{c}{\g12 runs 56595--56607, 56648--57323 } \\

\hline

bit & definition & L2 multiplicity\footnote{Level 2 triggering was turned off on all bits for runs 56605, 56607 and 56647.} & prescale \\

\hline

1 & \abbr{MORA}$\cdot$(\abbr{ST}$\cdot$\abbr{TOF}) & 1 & 1000/300\footnote{Prescaling for bits 1 and 4 were 1000 for runs prior to 56668 at which point they both were changed to 300.} \\
2 & \abbr{MORA}$\cdot$(\abbr{ST}$\cdot$\abbr{TOF})$\times$2 & 2/--\footnote{Level 2 triggering of bit 2 was set to 2 for runs prior to 56665 at which point it was turned off.} & 1 \\
3 & \abbr{MORB}$\cdot$(\abbr{ST}$\cdot$\abbr{TOF})$\times$2 & 2 & 1 \\
4 & \abbr{ST}$\cdot$\abbr{TOF} & 1 & 1000/300 \\
5 & (\abbr{ST}$\cdot$\abbr{TOF})$\cdot$\abbr{EC}$\times$2 & 1 & 1 \\
6 & (\abbr{ST}$\cdot$\abbr{TOF})$\cdot$(\abbr{EC}$\cdot$\abbr{CC}) & 2 & 1 \\
7 & \abbr{MORA}$\cdot$(\abbr{ST}$\cdot$\abbr{TOF})$\cdot$(\abbr{EC}$\cdot$\abbr{CC}) & -- & 1 \\
8 & \abbr{MORA}$\cdot$(\abbr{ST}$\cdot$\abbr{TOF})$\times$2 & -- & 1 \\
11 & (\abbr{EC}$\cdot$\abbr{CC})$\times$2 & -- & 1 \\
12 & (\abbr{ST}$\cdot$\abbr{TOF})$\times$3 & -- & 1 \\

\hline \hline

\end{tabular}

\end{singlespacing}
\end{center}
\end{minipage}
\end{table}
\vspace{20pt}
 % label: tab:data.trig.conf.2
%		
%		\begin{table}
\begin{minipage}{\textwidth}
\begin{center}
\begin{singlespacing}

\caption[Trigger Configuration for Single-sector Runs]{\label{tab:data.trig.conf.3}Trigger configuration for the single-prong runs of \g12. Trigger bits 7--12 were not used for these runs. \vspace{0.75mm}}

\begin{tabular}{cccc}

\hline

bit & definition & L2 multiplicity & prescale \\

\hline

1 & \abbr{MORA}$\cdot$(\abbr{ST}$\cdot$\abbr{TOF})$_{1}$ & sector 1 & 1 \\
2 & \abbr{MORA}$\cdot$(\abbr{ST}$\cdot$\abbr{TOF})$_{2}$ & sector 2 & 1 \\
3 & \abbr{MORA}$\cdot$(\abbr{ST}$\cdot$\abbr{TOF})$_{3}$ & sector 3 & 1 \\
4 & \abbr{MORA}$\cdot$(\abbr{ST}$\cdot$\abbr{TOF})$_{4}$ & sector 4 & 1 \\
5 & \abbr{MORA}$\cdot$(\abbr{ST}$\cdot$\abbr{TOF})$_{5}$ & sector 5 & 1 \\
6 & \abbr{MORA}$\cdot$(\abbr{ST}$\cdot$\abbr{TOF})$_{6}$ & sector 6 & 1 \\

\hline \hline

\end{tabular}

\end{singlespacing}
\end{center}
\end{minipage}
\end{table}
\vspace{20pt}
 % label: tab:data.trig.conf.3
%		
%		\begin{table}
\begin{center}
\begin{singlespacing}

\caption[Trigger Configuration (Tagger)]{\label{tab:data.trig.mor}Master-\abbr{OR} definitions for \g12. The \abbr{TDC} counters were used in the trigger and since each of these corresponds to several energy paddles, the energies given here are approximate. $T$-counter number 1 corresponds to the highest energy photon of approximately 5.4~GeV. Both \abbr{MORA} and \abbr{MORB} are referenced in terms of the trigger logic in Tables~\ref{tab:data.trig.conf.1}, \ref{tab:data.trig.conf.2} and \ref{tab:data.trig.conf.3}. The single-prong runs are listed in Table~\ref{tab:data.cook.singlesecruns}.\vspace{0.75mm}}

\begin{tabular}{c|cc|cc}

\hline

          & \multicolumn{2}{c|}{\abbr{MORA}} & \multicolumn{2}{c}{\abbr{MORB}} \\
run range & $T$-counters & energy (GeV)     & $T$-counters & energy (GeV) \\

\hline

56363--56400 & 1--47 & 1.7--5.4 & -- & -- \\
56401--56518 & 1--25 & 3.6--5.4 & -- & -- \\
56519--57323 & 1--19 & 4.4--5.4 & 20--25 & 3.6--4.4 \\

\hline

\emph{single-sector} & 1--31 & 3.0--5.4 & -- & -- \\

\hline \hline

\end{tabular}

\end{singlespacing}
\end{center}
\end{table}
\vspace{20pt} % label:  tab:data.trig.mor
%		
		\begin{table}
\begin{center}
\begin{singlespacing}

\caption[\abbr{EC} and \abbr{CC} Trigger Thresholds]{\label{tab:data.ecccthresh}Threshold values for the electromagnetic calorimeter (\abbr{EC}) and Cherenkov counter (\abbr{CC}) during the \g12 running period. \abbr{EC} thresholds are shown as \emph{inner}/\emph{total}, and \abbr{CC} thresholds are shown as \emph{left}/\emph{right}.\vspace{0.75mm}}

\begin{tabular}{cc|c}
\hline

\multicolumn{2}{c|}{\abbr{EC}} & \abbr{CC} \\

\emph{``photon"} & \emph{``electron"} \\


\hline

50/100~mV & 60/80~mV & 20/20~mV \\
150/300~MeV & 180/240~MeV & $\sim$0.4~photo-electrons \\

\hline \hline

\end{tabular}

\end{singlespacing}
\end{center}
\end{table}
\vspace{20pt} % label: tab:data.ecccthresh
		