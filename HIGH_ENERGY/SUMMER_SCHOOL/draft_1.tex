\documentclass[%
 reprint,
showpacs,
 amsmath,amssymb,
 aps,
]{revtex4-1}

\usepackage{graphicx}% Include figure files
\usepackage{dcolumn}% Align table columns on decimal point
\usepackage{bm}% bold math
\usepackage{subfig}

\begin{document}
	\preprint{APS/123-QED}
	\title{Photoproduction of the $\pi^0$ meson from 3.6 - 5.5~GeV }% Force line breaks with \\
	\author{Michael C. Kunkel}
	\affiliation{%
	 Forschungszentrum J\"ulich \\
	}%
	\collaboration{CLAS Collaboration}
	\date{\today}
	\begin{abstract}
		Exclusive neutral pion photoproduction ($\gamma p \rightarrow p \pi^0$) was measured in the CLAS detector at the Thomas Jefferson National Facility. The experiment employed a 1.1-5.5~GeV bremsstrahlung photon beam from 5.6~GeV electron beam created in the Continuous Electron Beam Accelerator Facility (CEBAF). The photon beam energy was impinged on a liquid hydrogen target. The neutral pions were detected via external conversion, $\pi^0 \rightarrow \gamma \gamma \rightarrow e^+e^-\gamma$, and subsequent Dalitz decay, $\pi^0 \rightarrow \gamma^* \gamma \rightarrow e^+e^-\gamma$. Measured differential cross-sections, $\frac{d\sigma}{dt}$ and $\frac{d\sigma}{d\cos \theta}$ are compared with the Regge model. The Regge theoretical calculations underestimate the differential cross sections between 3.9 and 4.6~GeV, but agree with data at photon energies 4.6-5.4~GeV.
	\end{abstract}
	\maketitle
\end{document}