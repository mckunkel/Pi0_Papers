% ****** Start of file apssamp.tex ******
%
%   This file is part of the APS files in the REVTeX 4.1 distribution.
%   Version 4.1r of REVTeX, August 2010
%
%   Copyright (c) 2009, 2010 The American Physical Society.
%
%   See the REVTeX 4 README file for restrictions and more information.
%
% TeX'ing this file requires that you have AMS-LaTeX 2.0 installed
% as well as the rest of the prerequisites for REVTeX 4.1
%
% See the REVTeX 4 README file
% It also requires running BibTeX. The commands are as follows:
%
%  1)  latex apssamp.tex
%  2)  bibtex apssamp
%  3)  latex apssamp.tex
%  4)  latex apssamp.tex
%
\documentclass[%
 reprint,
%superscriptaddress,
%groupedaddress,
%unsortedaddress,
%runinaddress,
%frontmatterverbose, 
%preprint,
showpacs,
%,preprintnumbers,
%nofootinbib,
%nobibnotes,
%bibnotes,
 amsmath,amssymb,
 aps,
%pra,
%prb,
%rmp,
%prstab,
%prstper,
%floatfix,
]{revtex4-1}

\usepackage{graphicx}% Include figure files
\usepackage{dcolumn}% Align table columns on decimal point
\usepackage{bm}% bold math
\usepackage{subfig}

%\usepackage{hyperref}% add hypertext capabilities
%\usepackage[mathlines]{lineno}% Enable numbering of text and display math
%\linenumbers\relax % Commence numbering lines
%
%\usepackage[showframe,%Uncomment any one of the following lines to test 
%%scale=0.7, marginratio={1:1, 2:3}, ignoreall,% default settings
%%text={7in,10in},centering,
%%margin=1.5in,
%%total={6.5in,8.75in}, top=1.2in, left=0.9in, includefoot,
%%height=10in,a5paper,hmargin={3cm,0.8in},
%]{geometry}

\begin{document}
\preprint{APS/123-QED}

\title{Photoproduction of the $\pi^0$ meson from 3.6 - 5.5~GeV }% Force line breaks with \\
%\thanks{A footnote to the article title}%

\author{Michael C. Kunkel}
 %\altaffiliation[Also at ]{Physics Department, Old Dominion University.}%Lines break automatically or can be forced with \\
%\author{Second Author}%
% \email{Second.Author@institution.edu}
\affiliation{%
 Forschungszentrum J\"ulich \\
% This line break forced with \textbackslash\textbackslash
}%

\collaboration{CLAS Collaboration}%\noaffiliation

%\author{Charlie Author}
% \homepage{http://www.Second.institution.edu/~Charlie.Author}
%\affiliation{
% Second institution and/or address\\
% This line break forced% with \\
%}%
%\affiliation{
% Third institution, the second for Charlie Author
%}%
%\author{Delta Author}
%\affiliation{%
% Authors' institution and/or address\\
% This line break forced with \textbackslash\textbackslash
%}%
%
%\collaboration{CLEO Collaboration}%\noaffiliation

\date{\today}% It is always \today, today,
             %  but any date may be explicitly specified

\begin{abstract}
Exclusive neutral pion photoproduction ($\gamma p \rightarrow p \pi^0$) was measured in the CLAS detector at the Thomas Jefferson National Facility. The experiment employed a 1.1-5.5~GeV bremsstrahlung photon beam from 5.6~GeV electron beam created in the Continuous Electron Beam Accelerator Facility (CEBAF). The photon beam energy was impinged on a liquid hydrogen target. The neutral pions were detected via external conversion, $\pi^0 \rightarrow \gamma \gamma \rightarrow e^+e^-\gamma$, and subsequent Dalitz decay, $\pi^0 \rightarrow \gamma^* \gamma \rightarrow e^+e^-\gamma$. Measured differential cross-sections, $\frac{d\sigma}{dt}$ and $\frac{d\sigma}{d\cos \theta}$ are compared with the Regge model. The Regge theoretical calculations underestimate the differential cross sections between 3.9 and 4.6~GeV, but agree with data at photon energies 4.6-5.4~GeV.

%\begin{description}
%\item[Usage]
%Secondary publications and information retrieval purposes.
%\item
%\pacs{13.20.Cz, 13.25.Cq, 13.30.Ce,  13.60.Le, 14.40.0n}
%%\pacs{PACS numbers: 13.20.Cz, 13.25.Cq, 13.30.Ce,  13.60.Le, 14.40.0n}% PACS, the Physics and Astronomy
%May be entered using the \verb+\pacs{#1}+ command.
%\item[Structure]
%You may use the \texttt{description} environment to structure your abstract;
%use the optional argument of the \verb+\item+ command to give the category of each item. 
%\end{description}
\end{abstract}


                             % Classification Scheme.
%\keywords{Suggested keywords}%Use showkeys class option if keyword
                              %display desired
\maketitle

%\tableofcontents

%
% ****** End of file apssamp.tex ******
\end{document}